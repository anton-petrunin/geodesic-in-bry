\documentclass[a4paper,10pt]{article}
\usepackage{paper}
\hypersetup{pdftitle={About all convex sets in
most of Riemannian manifolds}%
,pdfauthor={Alexander Lytchak and Anton Petrunin}
}
\begin{document}
 
\title{About all convex sets in\\
most of Riemannian manifolds}
\author{Alexander Lytchak and Anton Petrunin}
\date{}
\maketitle

\begin{abstract}
We give a necessary condition on a geodesic in a Riemannian manifold that can run in some convex hypersurface.
As a corollary we obtain a number of peculiar properties of \emph{all} convex sets that hold true in \emph{generic} Riemannian manifolds $(M,g)$.
In particular, we show that the set of non-extreme points on the boundary of any convex set in $(M,g)$
is a union of a countable set of geodesics.
\end{abstract}

\section{Introduction}
Let $\mathfrak{C}$ be the convex hull of a subset $A$ in a Euclidean space.
By Carathéodory's theorem, any point in $\mathfrak{C}\setminus A$ is an inner point of a geodesic contained in $\mathfrak{C}$;
that is, the complement $\mathfrak{C} \setminus A$ does not contain extreme points of $\mathfrak{C}$. 
{\color{red} Moreover, the convex hull of any compact subset
	is  compact.\footnote{We shoul add some statement like this, otherwise the Corollary 1.2 does not make much sense}}  
These well-known statements admit straightforward generalizations to the sphere and the Lobachevskian space. 
{\color{red} These statements are also easily seen to hold locally in any surface.} 

%to prove 
%in all simply connected non-positively curved surfaces;
%morover, locally it holds true in any two-dimensional Riemannian manifold.

Recall that a set $\mathfrak{C}$ in a Riemannian manifold $(M,g)$ is called \emph{convex} if for any pair of points $x,y\in \mathfrak{C}$ any minimizing geodesic $[x,y]$ lies in $\mathfrak{C}$.
A point in $\mathfrak{C}$ is called \emph{extreme} if it does not lie in an interior of a geodesic in~$\mathfrak{C}$.

It seems to be a folklore belief that a version of this statement should hold true in all Riemannian manifolds. 
For instance, it is claimed by Mohammad Ghomi and Joel Spruck \cite[Lemma 9.1]{Ghomi}.
In the present note we prove that the somewhat counter-intuitive opposite is the case for \emph{generic} Riemannian manifolds. 

{\color{red}  A subset of a topological space is \emph{meager} if it is a countable union
	of closed nowhere dense subsets.   
	
	
	Let $k$ be a natural number.
% Following \cite{eliashberg-mishachev},
 A \emph{$\mathcal C^k$-generic Riemannian metric $g$ on a manifold $M$ 
has a property $P$} if the set   of metric tensors $g$ that do not  satisfy $P$ 
is meager  in the space  of all smooth Riemannian metrics on $M$ equipped  with the  $\mathcal C^k$-topology.
Since the space of Riemannian metrics on $M$ is \emph{Baire}, this implies that the set
of metric tensors that do satisfy $P$ is dense in the space of all  Riemannian metrics.  }


% any set of $\mathcal C^k$-generic metric is dense  in the space of all Riemannian metric 
% with the  $\mathcal C^k$-topology.
%the complement of any meager subset is dense with  \emph{most} of Riemannian metrics are generic.


\begin{thm}{Theorem}\label{thm:main}
Let $M$ be a smooth manifold with dimension at least 3. \footnote{We can skip the assumption on the dimension. In dimension 1,2 the statement is obviously true}

Then, for $\mathcal C^2$-generic smooth Riemannian metric $g$ on $M$,
any closed convex set $\mathfrak{C}\subset (M,g)$ that is not a subset of a geodesic
satisfies the following properties:

\begin{subthm}{}
The set $\mathfrak{C}$ has nonempty interior.
\end{subthm}

\begin{subthm}{}
The set of all non-extreme points $\partial\mathfrak{C}$ {\color{red} is an at most countable union of non-interesecting geodesics of$(M,g)$}.
%can be covered by a countable set of geodesics of $(M,g)$.
In particular, the set of
 extreme points of $\mathfrak{C}$ is dense in~$\partial\mathfrak{C}$.
\end{subthm}

\end{thm}

{\color{red}
Statement (a) is essentially known: in dimension at least $4$ it is proved in \cite{Wilhelm}
and in dimension $3$ it is sketched in  \cite{Bryant}.}

%Note that 
The theorem implies that an analog of Carathéodory theorem does not hold in most  Riemannian manifolds.
Namely, we get the following corollary:

\begin{thm}{Corollary}\label{cor:caratheodory}
Suppose that $g$ is a $\mathcal C^2$-generic complete Riemannian metric on a smooth connected manifold $M$ with dimension at least 3.
{\color{red} Let $A$ be any compact 
%non-convex
  subset 
of $(M,g)$,   {\color{red} that is not contained in a minimizing  geodesic and does not contain
	the boundary of a convex set in $(M,g)$.}

Then the convex hull of $A$ is either equal to $M$ or it is not closed in $M$.}
%Define  $A_k$ inductively as  the union of all minimizing geodesics connecting pairs of  points in 
%$A_{k-1}$.
%Then, either $A_k=M$ for some $k$, or $A_{k+1}\supsetneq A_k$ for all $k$;
%the latter means that $A_k$ is not convex for any $k$.
\end{thm}

{\color{red} The convex hull in the above Corollary is the smallest convex  subset of $M$ that contains $A$.  The assumption on  $A$ is automatically  satisfied, for instance, if $A$ has dimension (Hausdorff or topological) strictly between $1$ and $n-1$.}


The proofs of these statements are built on the key lemma stated in the following section; 
it describes a necessary condition on a geodesic in a Riemannian manifold that can run in some convex{\color{red}, possibly non-smooth, hypersurface.   If the tangent cone to the convex hypersurface at a point  of the geodesic contains a plane, this condition implies a non-trivial
statement on the curvaure operator of the manifold.
The remaining part is   not surprising:  it shows that the curvature operators of  most Riemannian manifolds do not have this property. However,  the formal proofs 
%nearly self-evident;
%The
%its formal proofs use
relying on 
 Thom's transversality theorem are slightly technical.}
 % and it takes the remaining sections.

\section{Key lemma}
Let $(M,g)$ be a Riemannian manifold.
%we will often use shortcut $\langle \vec{v},\vec{w}\rangle$ for $g(\vec{v},\vec{w})$.
Let $\mathfrak{C}$ be a closed convex set in $(M,g)$.
Given a point $x\in \mathfrak{C}$, 
%define its 
{\color{red} the tangent cone $\K_x\mathfrak{C}\subset \T_x$ is  the closure of the set of all velocity vectors of geodesics that start at $x$ and run in $\mathfrak{C}$.  
The tanegnt cone $\K_x\mathfrak{C}$ is a convex cone in $\T_x$ and it coinicdes with 
$\T_x$  if and only 
%Note that 
$x$ lies in the interior of $\mathfrak{C}$, see, for instance, \cite{}.}
% if and only if $\K_x\mathfrak{C}=\T_x$.

Given a tangent vector $\vec{x}\in\T_pM$, set
\[R(\vec{x})\:\vec{v}\mapsto \Rm(\vec{v},\vec{x})\vec{x},\]
where $\Rm$ denotes the curvature tensor of $(M,g)$.
Note that $R(\vec{x})\:\T_p\to \T_p$ is  self-adjoint.
The Jacobi equation along a geodesic $\gamma$ can be written as 
\[\vec{i}''+R(\gamma')\cdot \vec{i}=0,\]
here  $\vec{i}''$ is a shortcut for $\nabla^2_{\gamma'}\vec{i}$.
 
\begin{thm}{Key lemma}\label{lem:key}
Let $(M,g)$ be a Riemannian manifold and {\color{red}  $\gamma :(a_0,b_0)\to M $ be a geodesic %in $(M,g)$ that is defined on an open interval.
contianed} in a closed convex set $\mathfrak{C}\subset (M,g)$.
Then the tangent cone of $\mathfrak{C}$ is parallel along $\gamma$; that is, the parallel translation along $\gamma$ defines a bijection between the tangent cones $\K_{\gamma(a)}\mathfrak{C}$ and $\K_{\gamma(b)}\mathfrak{C}$ for any $a,b \in (a_0,b_0)$.

Moreover
\begin{subthm}{}    \footnote{Maybe make (1) and (2) instead of (a) and (b)? Or change the notation of $a$ and $b$ for times on $\gamma$}
For any $\vec{v}\in \K_{\gamma(a)}\mathfrak{C}$ we have
\[R(\gamma'(a))\cdot \vec{v}\in \K_\vec{v}[\K_{\gamma(a)}\mathfrak{C}].\]
\end{subthm}

\begin{subthm}{} Let $L(a)$ be the maximal linear subspace of $\K_{\gamma(a)}\mathfrak{C}$.
Then $L(a)$ is an invariant subspace of $R(\gamma'(a))$ for any $a$.
\end{subthm}

\end{thm}

%The proof is based on the next observation;
%it follows immediately from the fact that Jacobi fields describe the difference between infinitesimally close geodesics.

%\begin{thm}{Observation}
%Let  $\mathfrak{C}$ be a closed convex subset in a Riemannian manifold $(M,g)$.
%Suppose $\vec{i}$ is a Jacobi field along a geodesic $\gamma\:[a,b]\to \mathfrak{C}$ in $(M,g)$.
%Assume that $\vec{i}(a)\in \K_{\gamma(a)}\mathfrak{C}$ and $\vec{i}(b)\in %\K_{\gamma(b)}\mathfrak{C}$.
%Then $\vec{i}(t)\in \K_{\gamma(t)}\mathfrak{C}$ for any $t\in [a,b]$.
%\end{thm}

\parit{Proof of \ref{lem:key}.} 
{\color{red}  All statements is local.  Thus we may replace $M$ by a small ball around a point. By doing so we may assume that any pair of points of $M$ are connected by a unique geodesic and that no conjugate points exist in $M$.  In particular, for any $a_0<t_1<t_2<b_0$ and any tangent vectors $v_i \in \T_{\gamma (t_i)}$ there exists exactly one Jacobi field $I$ along $\gamma$ with
	$I(t_i)=v_i$.



Since Jacobi fields are by definition variational fields of geodesic variations, the convexity of $\mathfrak{C}$ immediatly implies: 

\begin{thm}{Observation}
%	Let  $\mathfrak{C}$ be a closed convex subset in a Riemannian manifold $(M,g)$.
	Suppose $\vec{i}$ is a Jacobi field along% a geodesic
	 $\gamma$ and $a_0<a<t<b<b_0$. If 
	% \:[a,b]\to \mathfrak{C}$ in $(M,g)$.
	%Assume that
	 $\vec{i}(a)\in \K_{\gamma(a)}\mathfrak{C}$ and $\vec{i}(b)\in \K_{\gamma(b)}\mathfrak{C}$
	then $\vec{i}(t)\in \K_{\gamma(t)}\mathfrak{C}$.
	% for any $t\in [a,b]$.
\end{thm}


%Choose a subinterval 
 The statement that the tangent cones $\K_{\gamma(a)}\mathfrak{C}$  are parallel along $\gamma$ follows from the fact that the parallel translation can be defined via geodesics,
	see \cite{Ber-Nik}[Section 13], \cite{Petruninpar}. For the convenience of the reader we include a short argument.
	
	
	 Fix  $[a,b] \subset (a_0,b_0)$.}
% in the domain of $\gamma$.
Given a large positive integer $k$, consider arithmetic progression
$(t_i)$ such that $t_0=a$, $t_k=b$.

Choose a tangent vector $v_0\in\T_{\gamma(a)}$. 
Consider the sequence of vectors $v_i \in\T_{\gamma(t_i)}$ defined recursively by $v_{i+1}=2\cdot \vec{i}_i(t_{i+1})$, where $t\mapsto \vec{i}_i(t)$ is the Jacobi field along $\gamma$ such that $\vec{i}_i(t_i)=v_i$ and $\vec{i}_i(t_{i+2})=0$.
%since $k$ is large, the mentioned Jacobi fields are uniquely defined.

Define $\iota_k\:\T_{\gamma(a)}\to \T_{\gamma(b)}$ by setting $\iota_k(\vec{i}_0)\df \vec{i}_k$.
According to the observation, if $\vec{i}_0\in \K_{\gamma(a)}$, then $\iota_k(\vec{i}_0)\in \K_{\gamma(b)}$.
Note that \footnote{Can one cite something besides Nikolaev?  He referes to Cartan} $\iota_k(\vec{i}_0)$ converges to the parallel translation of $\vec{i}_0$ along $\gamma$ as $k\to \infty$.
It follows that parallel translation along $\gamma$ maps $\K_{\gamma(a)}\mathfrak{C}$ in $\K_{\gamma(b)}\mathfrak{C}$.
Swapping $a$ and $b$, we get the opposite inclusion.
That is, the tangent cone $K_{\gamma(t)}\mathfrak{C}$ is parallel along $\gamma$.

Let us use parallel translation along $\gamma$ to identify the tangent spaces at the points on $\gamma$.
This way we will identify the tangent cones $\K_{\gamma(t)}\mathfrak{C}$ for all $t$;
denote the obtained cone by $\K_\gamma\mathfrak{C}$.

For $v\in \K_\gamma\mathfrak{C}  = \K_{\gamma  (a)}\mathfrak{C}$ and small $\epsilon >0$ consider the unique Jacobi field   $\vec{i} ^{\epsilon}$ along $\gamma$ with
%  \in K_\gamma\mathfrak{C}$ and $\eps>0$;
%consider the Jacobi field $\vec{i}$ such that
 $\vec{i}^{\epsilon} (a+\eps)\z=\vec{i} ^{\epsilon}(a-\eps)=v$.
Due to the  Jacobi equation,
% we have
\[\vec{i}^{\epsilon} (a)=v +\eps^2\cdot R(\gamma'(a))\cdot v +o(\eps^2).\]
According to the observation, $\vec{i} ^{\epsilon}(a)\in \K_{\gamma} \mathfrak{C}$ for any $\eps>0$.
Since $\K_{\gamma} \mathfrak{C}$ is a closed convex cone, we get $R(\gamma')\cdot v\z\in \K_v[\K_\gamma\mathfrak{C}]$, proving (a).

Finally note that $v\in L(a)$ if and only if $\K_v[\K_{\gamma(a)}\mathfrak{C}]=\K_{\gamma(a)}\mathfrak{C}$.
Therefore, if $v\in L(a)$, then $\pm R(\gamma'(a))\cdot v\in \K_{\gamma(a)}\mathfrak{C}$ and hence $R(\gamma'(a))\cdot v\in L(a)$.
That is, $L(a)$ is invariant subspace of $ R(\gamma'(a))$, for any $a$.
\qeds
  
  
  
  \section{Reduction to a technical statement}
 Let $M$ be a smooth manifold.  For any $\epsilon >0$ consider the set $G^{\epsilon}$ of all smooth Riemannian metrics  $g$ on $M$ satisfying the following property. There exists 
 a geodesic
 $\gamma$ in $(M,g)$ of length at least $\epsilon$   and a family $V_t \subset \T_{\gamma (t)}$ of  $k$-dimensional linear subspaces, such that $V_t$ is parallel along $\gamma$ and such that
 $R(\gamma'(t))  (V_t)\subset V_t$  for all $t$.
 
The  $G^{\epsilon}$ is closed with respect to the $\mathcal C^2$-topology, since the  geodesics and the curvature tensor depend continuously on the Riemannian metric in this topology.
    
  The proof of following result will be  given in the next section:
  \begin{thm}  {Proposition}\label{technic}
  	For any smooth manifold of dimension $m\geq 3$ and any $\epsilon >0$ the set $G^{\epsilon}$ is nowhere dense.
  	\end{thm}
  
  
  From the Key Lemma and this Proposition we can readily deduce the main results of the paper. 
  
 
 \parit{Proof of the main theorem (\ref{thm:main}).}
   Set $G= \cup _n G^{\frac 1 n}$, where the set $G^{\epsilon}$ of Riemannian metrics 
   is defined as above.  By Proposition \ref{technic}, the set $G$ is meager.
   
   Let $g$ be any Riemannian metric not contained in $G$.   
   Then, for any 
   geodesic $\gamma$ in $(M,g)$ and for any $1< k< m =\dim (M)$,  there do not exist 
    $k$-dimensional vector spaces  $V_t \subset \T _{\gamma(t) } $   parallel  along $\gamma$,
    such that   $R(\gamma'(t))  (V_t)\subset V_t$,  for all $t$.
    
    Due to  the Key Lemma, for any point $x\in \mathfrak C$ that is not extrem in $\mathfrak C$, the tangent cone $\K_{\gamma (t)} \mathfrak C$ is either the whole tangent space $\T_{\gamma (t)}$ or it does not contain any plane. 
    
  %   geodesic $\gamma :(a,b)\to M$ contained in 
  % the closed convex set  $\mathfrak C$, the tangent cone $\K_{\gamma (t)} \mathfrak C$ is either equal to the whole tangent space $\T_{\gamma (t)}$ or does not contain any plane. 
    
 
 
 By the structure theory of convex sets in Riemannian manifolds, \cite{}, $\mathfrak C$ is homeomorphic to a topological manifold with boundary $(N, \partial N)$ of some dimension 
 $l$.  Moreover, $N\setminus \partial N$ is a totally geodesic submanifold on $M$,
 in particular, any point $p\in N\setminus \partial N$  is not extreme in $\mathfrak C$.
 
 The tangent cone  $\K_{p} \mathfrak C$ at any   $p \in N\setminus \partial N$ is an  $l$-dimensional linear space. Thus,   $l=1$ or $l=n$. In the first case, $N$ is a geodesic in $M$, hence $\mathfrak C$ is contained in a geodesic as well.
 In the latter case,  $N$ is open in $M$.   This proves (a).
 
 
 In order to prove (b), we observe that for any non-extreme point $x\in \partial  \mathfrak C$,
 the tangent cone $\K_x \mathfrak C$ splits off a line but not a plane.  Thus this line is unique  and is the tangent line of the (therefore unique) geodesic $\gamma$ contained in $\mathfrak C$ and having $p$ as an inner point.  
 
 We extend $\gamma$ to its maximal length, so it stays in $\mathfrak C$, and delete the endpoints of $\gamma$  (if they exist).  By the first statement of the Key Lemma, all points on $\gamma$ lie on $\partial \mathfrak C$. 
 
 
 By definition, all such geodesics consist of non-extreme points. As seen above, through each non-extreme point on $\partial \mathfrak C$ passes exactly one such geodesics.  Thus we have written the set of non-extreme points $E$  of $\mathfrak C$ contained in the boundary $\partial \mathfrak C$ in a unique way as a union of pairwise non-intersecting $(M,g)$-geodesics $\gamma_e$ 
 of positive length.  
 
 If there are more than countably many such geodesics $\gamma _e$ then the union  of all these geodesics is not countably rectifiable, contradicting to the countable rectifyibility of the set of points  
 $x\in \mathfrak C$ for which the tangent cone $\K_x \mathfrak C$ does not contain a line, \cite{}.
 
   
 
 
 
% Choose a geodesic $\gamma$ of $(M,g)$ defined on an open interval.
% Assume that $L(t)\subset K_{\gamma(t)}\mathfrak{C}$ is provided by the key lemma (\ref{lem:key}).
% Since the tangent cone $K_{\gamma(t)}\mathfrak{C}$ is parallel along $\gamma$, so is $L(t)$.
 
% By the key lemma, $L(t)$ is invariant subspace of $R(\gamma'(t))$ for any~$t$.
% Taking derivatives by $t$, we get that $L(t)$ is invariant subspace of $R^{(k)}(\gamma'(t))$ for any positive integer $k$.
 
% Evidently $L(t)\ni \gamma'(t)$.
% If $\gamma$ runs in the boundary $\partial\mathfrak{C}$, then $L(t)\subsetneq\T_{\gamma(t)}$.
% Since $g$ is generic,
% by \ref{prop:R'}, it follows that $L(t)$ is the 1-dimensional subspace spanned by $\gamma'(t)$.
 
 %TBC
 \qeds
 
 \parit{Proof of the main theorem (\ref{cor:caratheodory}.}
 The convex hull $\mathfrak C$ of $A$ can be obtained as the following infinite union.
 We start with $A=A_1$ and define inductively $A_{i+1}$ to be  the union of all minimizing geodesics between pairs of  points of $A_i$. Then  the union $\mathfrak K= \cup _{i=1}^{\infty} 
 A_i$ is the convex hull of $A$. 
 
 
 Assume now that  $\mathfrak K$ is  closed and not equal to $M$.
 Since $A$ is not contained in a geodesic, by our main theorem, $\mathfrak K$ has a non-trivial interior.   By the explicit construction of $\mathfrak K$ above, any point $x\in \mathfrak K \setminus A$ is not an extreme point of $\mathfrak K$.   
 
 By the main theorem the topological manifold $\partial \mathfrak K$ is the union of the compact subset $A\cap \partial \mathfrak K$ and an at most countable union of geodesics.  But the 
 (by assumption non-empty)  $(n-1)$-dimensional topological manifold $\partial \mathfrak K \setminus A\cap \partial \mathfrak K$ has Hausdorff dimension $n-1$ and  cannot be a union of countably many geodesics.
 
 
 This contradiction finishes the proof.
 \qeds
 
 
 
 \section{Genericity arguments}
 \subsection{Higher Jacobi operators}
 For a Riemannian metric $g$ on $M$, and a tangent vector $x\in \T_p M$ we 
 consider as before $R=R(\vec{x})\:\vec{v}\mapsto \Rm(\vec{v},\vec{x})\vec{x}$.
 We define now for all natural $k$, the symmetric endomorphism
 \[R^{(k)}=R^{(k)}(\vec{x})\df\nabla^k_\vec{x}R(\vec{x}).\]
 of the tangent space $\T _p $.
 

For any $k$, the endomorphism $R^{(k)} (x)$ sends the vector $x$ to $0$, thus it preserves the 
 orthogonal complement $x^{\perp}$.
 
 
 
 Let now $G^{\epsilon}$ be defined as before and let $\gamma$ be  geodesic starting at $p$ in the direction $x$. Let further $V_t$ be the field of $k$-dimensional linear spaces parallel along $\gamma$ and satisfying $R(\gamma'(t)) (V_t)\subset V_t$.
 
 Then, taking covarinat derivatives of this inclusion, we see that the vector space $V_0\subset \T_p M$ satisfies $R^{(k)} (V_0 \subset V_0)$ for all $k$.   Then also the interscetion of $x^{\perp}$  with $V_0$ is invariant under all higher Jacobi operators $R^{(k)}$.
 
 
 We define $N^{k}$ to be the set of all smoot Riemannian metrics, such that there exists a non-zero tangent vector $x$ at some point $T_p M$ and a subspace $V\subset x{\perp}$ of dimension $0<\dim (V)<n-1$ such that $R^{(j)} (V) \subset V$ for all $j=0,...,k$.
 
 Note that for $k=0$ the condition is always fulfilled, so $N^0$ is just the set of all metrics.
 It can be seen that $N^1$ is rather large as well, however, it seems plausible that $N^k$ is a rather "thin" set of metric for large $k$. This will be made precise below.
 
 
 Above we have observed that $G^{\epsilon}$ is contained in the set $N^{k}$, for all natural $k$.
 Clearly, in the definition of $N^{k}$ we may restrict ourselves to unit vectors $x$.
 
 Finally, by the very definition, $N^k$ is closed with respect to $\mathcal C^{k+2}$-topology in the space of all metrics.
  
  
  Thus, the following result which will be proved below directly implies Proposition \ref{technic}:
  
  \begin{thm}{Proposition}
  Let $M$  be a smooth manifold of dimension at leats $3$. Then, for $k\geq 6$,   the set $N^k$ is nowhere dense in the set of all Riemannian metrics.	
  	\end{thm}
 
 
  
 
 
 
  
\section{Taylor expansion of metric tensor}\label{sec:jet}

In this section we calculate the Taylor expansion of metric tensor in the normal coordinates.
This formula was derived by Old\v{r}ich Kowalski and Martin Belger \cite[Proposition 2.2]{kowalski-belger} (that is the earliest reference we have found),
but we present calculations since some notations will be used further.

Let $(M,g)$ be a Riemannian manifold.
Recall that $\Rm$ denotes the curvature tensor.

Choose a tangent vector $\vec{x}_p\in \T_pM$; 
extend it to a parallel vector field $\vec{x}$ in the normal coordinates at $p$.
That is, set 
\[\vec{x}_{\exp_p\vec{w}}=(d_\vec{w}\exp_p)\vec{x}_p\] for all small $\vec{w}\in \T_p$.
Note that $(\nabla^k_\vec{x} \vec{x})_p=0$ for any $k\ge 1$.

Given a tensor field $S$, let us use the shortcut $S'$  for $\nabla_\vec{x}S$ at $p$.
Recall that $R=R(\vec{x})\:\vec{v}\mapsto \Rm(\vec{v},\vec{x})\vec{x}$.
Therefore
\[R^{(k)}=R^{(k)}(\vec{x})\df\nabla^k_\vec{x}R(\vec{x}).\]
Note that 
\begin{itemize}
\item $R^{(k)}(\vec{x})\:\T_p\to\T_p$ is a self-adjoint operator, 
\item $\vec{x}\perp R^{(k)}(\vec{x})\cdot \vec{v}$ for any vector field $\vec{v}$, and
\item $\vec{x}\mapsto R^{(k)}(\vec{x})$ is a homogenius polynomial of degree $k+2$ defined for all $\vec{x}\in \T_p$.
\end{itemize}

 


\begin{thm}{Lemma}\label{lem:jacobi}
For any tangent vector $\vec{x}$ at a point $p$ of a Riemannian manifold manifold $(M,g)$ and any integer $n$ there are three self-adjoint operators 
\[A_n=A_n(\vec{x}),\ B_n=B_n(\vec{x}),\ C_n=C_n(\vec{x})\:\T_p\to \T_p\]
that satisfy the following conditions:

\begin{subthm}{lem:jacobi:ABC}
For any Jacobi field $\vec{i}$ along the geodesic $\gamma\:t\mapsto \exp_p(t\cdot \vec{x})$ the following identity holds:
\[\langle \vec{i},\vec{i}\rangle^{(n)}
=
\langle A_n\cdot  \vec{i},\vec{i}\rangle+ \langle B_n\cdot \vec{i}, \vec{i}'\rangle+\langle C_n\cdot \vec{i}', \vec{i}'\rangle.\]
\end{subthm}

\begin{subthm}{lem:jacobi:R}
 $\vec{x}\mapsto A_{n}(\vec{x})$, $\vec{x}\mapsto B_{n+1}(\vec{x})$ and $\vec{x}\mapsto C_{n+2}(\vec{x})$ are homogeneous polynomials of degree $n$ defined for all $\vec{x}\in \T_p$.
\end{subthm}

\begin{subthm}{lem:jacobi:A-R}
The following differences  
\[A_{n+2}-2\cdot R^{(n)},
\quad 
B_{n+3}-4\cdot(n+2)\cdot R^{(n)},
\quad
C_{n+4}-2\cdot(n+1)\cdot(n+4)\cdot R^{(n)}\]
depend only on $R,R',\dots, R^{(n-2)}$.
\end{subthm}


\end{thm}

\parit{Proof.}
Applying recursively the Jacobi identity $\vec{i}''=-R\cdot \vec{i}$ for the geodesic $\gamma\:t\mapsto\exp_p(t\cdot \vec{x})$, we get that the following identity 
\[\langle \vec{i},\vec{i}\rangle^{(n)}
=
\langle A_n\cdot  \vec{i},\vec{i}\rangle+ \langle B_n\cdot \vec{i},\vec{i}'\rangle+\langle C_n\cdot \vec{i}',\vec{i}'\rangle\]
where $A_0=1$, $B_0=0$, $C_0=0$, and the operators $A_n$, $B_n$, and $C_n$ satisfy the following recursive relations:
\begin{align*}
A_{n+1}&=A_n'-\tfrac12\cdot R\cdot B_n-\tfrac12\cdot B_n\cdot R
\\
B_{n+1}&=B_n'+ 2\cdot A_n-C_n\cdot R-R\cdot C_n
\\
C_{n+1}&=C_n'+ B_n
\end{align*}

\renewcommand{\arraystretch}{1.5}
\begin{figure}[!ht]
\centering
\begin{tabular}{ l|c|c|c }
$n$ & $A$ & $B$ & $C$ \\ \hline
0& $1$   &  $0$  & $0$ \\ \hline
1& $0$   &  $2$  & $0$ \\ \hline 
2& $-2\cdot R$ & $0$ & $2$ \\ \hline 
3& $-2\cdot R'$ & $-8\cdot R$ & 0  \\ \hline 
4& $-2\cdot R''+8\cdot R^2$ & $-12\cdot R'$ & $-8\cdot R$  \\ \hline
5& $-2\cdot R^{(3)}+\dots$ 
& $-16\cdot R''+16\cdot R^2$ & $-20\cdot R'$  \\ \hline
6
&$-2\cdot R^{(4)}+\dots$
&$-20\cdot R^{(3)}+\dots$
&$-36\cdot R''+16\cdot R^2$
\\
\end{tabular}
\end{figure} 

It proves part \ref{SHORT.lem:jacobi:ABC};
by looking at the table, and applying induction, one gets~\ref{SHORT.lem:jacobi:R} and~\ref{SHORT.lem:jacobi:A-R}.\qeds

Following \cite{eliashberg-mishachev}, we will denote by $J^n_g(p)$ the $n$-jet of a metric tensor $g$ at the point $p$;
it cares information on all partial derivatives of $g$ at $p$ up to order~$n$ in some (and therefore every) local coordinate system at $p$.
Equivalently, the jet $J^n_g(p)$ can be described by Taylor expansion up to degree $n$ of $g$ at $p$ (in some, and therefore every, local coordinate system at $p$).

Assume $g$ is positive definite at $p$.
Note that $J^n_g(p)$ determines the polynomials 
$\vec{x}\mapsto R(\vec{x})$, $\vec{x}\mapsto R'(\vec{x}),\dots, \vec{x}\mapsto R^{(n-2)}(\vec{x})$ defined for all $\vec{x}\in \T_p$. 
In other words, the map 
\[\rho_n\:J^n_g(p)\mapsto (R,\dots,R^{(n-2)})\]
is defined.
The domain $\Dom \rho_n\subset J^n_g(p)$ of $\rho_n$ is the open set of all $n$-jets such that $g$ is positive definite at $p$.

\begin{thm}{Proposition}\label{prop:submersion}
For any jet $J^n_g(p)\z\in \Dom \rho_n$ the differential $d_{J^n_g(p)}\rho_n$ has a right inverse $s$.
Moreover $s$ can be chosen so that the jets in its image correspond to metric tensors that share the normal coordinates with $g$.

In particular, the map $\rho_n$ is a submersion.
\end{thm}

\parit{Proof.}
Choose a Riemannian metric $g$ in a neighborhood of $p$.
Let $C_i=C_i(\vec{x})$ be as in \ref{lem:jacobi}.

Choose a vector filed $\vec{v}$ that is parallel in the normal coordinates at~$p$;
that is, $\vec{v}_{\exp_p\vec{w}}=(d_\vec{w}\exp_p)\vec{v}_p$ for some $\vec{v}_p\in\T_p$ and all small $\vec{w}\in \T_p$.

Set $\vec{v}(t)=\vec{v}_{\exp_p(t\cdot \vec{x})}$ and $\vec{i}(t)=t\cdot \vec{v}(t)$.
Note that $\vec{i}$ is a Jacobi field along the geodesic $t\mapsto \exp_p(t\cdot \vec{x})$.
Observe that 
\[t^2\cdot \langle \vec{v}(t),
\vec{v}(t)\rangle=\langle \vec{i}(t), \vec{i}(t)\rangle,
\quad 
\vec{i}(0)=0,
\quad 
\text{and} \quad 
\vec{i}'(0)\z=\vec{v}_p.\]
Applying \ref{lem:jacobi}, we get
\[\langle\vec{i}(t), \vec{i}(t)\rangle=\sum_{i=0}^{n+2}\tfrac1{(n+2)!}\cdot\langle C_n(\vec{x})\cdot \vec{v}_p,\vec{v}_p\rangle\cdot t^{n+2}+o(t^{n+2}).\]
Applying the last identity for $t=1$ and arbitrary $\vec{x}$, we get the equality
\[\langle\vec{v}, \vec{v}\rangle=\sum_{i=2}^{n+2}\tfrac1{(n+2)!}\cdot\langle C_n(\vec{x})\cdot \vec{v}_p,\vec{v}_p\rangle+o(|\vec{x}|^{n})\]
at the point $\exp_p \vec{x}$.
It follows that in the normal coordinates at $p$,
the $n$-jet $J^n_g(p)$ is uniquely described by the polynomials
\[\vec{x}\mapsto C_2(\vec{x}),
\ \dots\ ,\  
\vec{x}\mapsto C_{n+2}(\vec{x}).\]

It remains to applying \ref{lem:jacobi:A-R} inductively on $n$.
\qeds 

\section{Genericity}

%\begin{thm}{Proposition}\label{prop:R'}
%Let $M$ be a smooth manifold of dimension $n\ge 3$.
%Then a generic Riemannian metric $g$ on $M$ satisfies the following property:
%for any tangent vector $\vec{x}\ne 0$ the operators
%$R(\vec{x}), R'(\vec{x}), \dots, R^{(4)}(\vec{x})$ have only two nontrivial common invariant %subspaces: the 1-dimensional space spanned by $\vec{x}$ and its orthogonal complement.
%\end{thm}

%it is true without assumption n\ge 3, but might be misleading

The proposition follows from the Thom transversality theorem \cite[2.3.2]{eliashberg-mishachev}, \ref{prop:submersion}, and the following lemma.


\begin{thm}{Lemma}
Let $\mathfrak{S}_{k,n}$ be the space of all $k$-tuples $A_1,\dots, A_k$ of self-adjoint operators $\RR^{n-1}\to\RR^{n-1}$.
Then the set of all $k$-tuples $(A_1,...,A_k) \in \mathfrak{S}_{k,n}$ having a common nontrivial invariant subspace is an algebraic subset of $\mathfrak{S}_{k,n}$ of codimension $(n-2)\cdot (k-1)$.
\end{thm}

\parit{Proof.}\qeds


\bibliographystyle{alpha}
\bibliography{convex}
\end{document}
