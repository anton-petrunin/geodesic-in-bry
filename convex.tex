\documentclass[a4paper,10pt]{article}
\usepackage{paper}
\hypersetup{pdftitle={Geodesics that run in convex surfaces}%
,pdfauthor={Alexander Lytchak and Anton Petrunin}
}
\begin{document}
 
\title{Geodesics that run in convex surfaces}
\author{Alexander Lytchak and Anton Petrunin}
\date{}
\maketitle

\begin{abstract}
We give a partial answer to the question ``When a geodesic in a Riemannian manifold can run in a convex hypersurface''.
As a corollary we get that any convex hypersurface $S$ in a generic Riemannian manifold $M$ contains at most countable set of geodesics.
In particular, if $\dim M\ge 3$ then most of the points on the boundary of a convex set $K\subset M$ are not interior points of geodesics in $K$.
\end{abstract}

\section{Introduction}
 Consider  a compact subset  $A$ in a Euclidean space $\R^n$, the convex hull $K$ of $A$
 is compact and any non-extremal point of $K$ is contained in $A$. Thus,  any point in $K\setminus A$ is an inner point of a geodesic contained in $K$. More specifically,
 if $K$ has a non-empty interior,  $\partial K \setminus A$ is  a union of geodesics.  
 These well-known statements follows from Carathéodory's theorem in convex geometry and generalize in a straightforward way to the sphere $\mathbb S^n$ and the hyperbolic space.
 
 
 The above statement is also easy to prove in all simply connected non-positively curved surfaces and locally in any two-dimensional Riemannian manifold.
 

It seems to be a folklore belief that a version of this statement should hold true in all Riemannian manifolds, for instance, it is claimed in \cite[Lemma 9.1]{Ghomi}.   In the present note we prove that the somewhat counter-intuitive opposite is the case for  \emph{most} Riemannian manifolds.

\begin{thm}{Key lemma}
Suppose that a geodesic $\gamma$ in a  Riemannian manifold $M$ runs in a convex hypersurface $S\subset M$.
Denote by $K_p$ the tangent cone of $S$ at $p\in\gamma$.
Then $K_p$ is parallel along $\gamma$; that is a parallel translation from $x$ to $y$ along $\gamma$ sends $K_x$ to $K_y$.

Moreover, for any $V\in K_p$ we have $\Rm(V,T)T\notin $

In particular, if $K_p$ contains a plane spanned by $T$ and $V$, then 
$\Rm(V,T,T,V)$


\end{thm}





In order to formulate the result we introduce some notation.   For a Riemannian manifold $(M,g)$ (always smooth in this note), and
a non-zero tangent vector $v\in T_xM$ we denote by $v^{\perp} \subset T_xM$ the orthogonal complement of the the line generated by $v$.
For natural number $k=0,1,...$  and a 
tangent vector $v\in T _xM$ we denote by \emph{higher Jacobi-operators in the direction of $v$} $R^k _v :v^{\perp}_v\to v^{\perp}$ defined inductively  as follows:  
$$R^k _v(w):=  \nabla _v (\nabla _v)...\nabla_v R (w,v)v \;.$$ 

  For $k\geq 1$, we say that a Riemannian manifold $(M,g)$ satisfies the (genericity) condition
$G_k$ if   for all points $x\in M$ and all non-zero tangent vectors $v\in Tx_M$ the symmetric endomorpshism
$R^0 _v, R^1_v, R^2 _v,...R^k_v$ have no common invariant subspaces different from $\{0\}$ and $v^{\perp}$ in $v^{\perp}$

We denote by $\mathcal G^k (M)$ the set of all Riemannian metrics on $M$ satisfying the condition $G^k$.
Clearly, the set $\mathcal G^k (M)$ is open in the set of all smooth Riemannian metrics with respect to the 
$\mathcal C^{k+2}$-topology.

 Not surprisingly $\mathcal G^k (M)$ is also dense for sufficiently large $k$ if the dimension of $M$ is at least $3$.
The value of $k$ is probably not optimal in the following statement, which we derive from general transversality considerations:

\begin{thm}{Proposition}
Let $M$ be an $n$-dimensional smooth manifold with $n\geq 3$.  Then the set $\mathcal G^6 (M)$ is dense in the 
$\mathcal C^{8}$ topology.  
\end{thm}

For $n>5$, the proof shows that  the value of $k$  can be decreased from $6$ to $3$, see Proposition \ref{prop:  } below. 
The main result in this paper is that all closed convex subsets for any Riemannian manifold in this larger set $\mathcal G^k$
behave counter-intuitively:


\begin{thm} \label{thm: main}
  Let $M$ be smooth   manifold of dimension at least $3$.  
		Let $g$ be  a generic  Riemannian metric on $M$ with respect to 
  the $\mathcal C^8$-topology.  Let $K\neq M$ be any closed convex subset of dimension larger than one in  the Riemannian manifold $(M,g)$
  
Then  $K$
% is a point, or  a geodesic, or the whole manifold $M$, or $K$ 
has  a non-empty interior and non-empty boundary $\partial K$.  Moreover, $\partial K$ contains a point $x$ with the following properties: 

 No geodesic contained in $\partial K$ contains the point $x$. In particular,  $x$ is not   an inner  point of any geodesic contained in $K$. 
 \end{thm} 

%In fact, we show that the set of Riemannian metrics satisfying the conclusion of the above theorem contains an open
%and dense set in the space of all smooth Riemannian metrics with respect to $\mathcal C^8$-topology.
  
In fact, we prove, that  \emph{most} points in $\partial K$ has the above property, namely all points for which the tangent space $T_xX$ is a Euclidean halfspace, see Theorem \ref{} below.  The last statement becomes clearer in dimension $3$:

\begin{thm}{Corollary}
  Let $g$ be a $\mathcal C^8$-generic Riemannian metric on any $3$-dimensional manifold $M$.
  Let $K\neq M$ be any closed convex subset of $M$ of dimension $>1$.  Then $\partial K$ is a topological surface and there at most a countable number of geodesics contained  in $\partial K$. 
\end{thm}     

 
In particular, this result implies that a $\mathcal C^8$-generic Riemannian metric on a $3$-dimensional manifold $M$ does not contain
totally geodesic surfaces, thus extending the main result if \cite[Theorem A]{Wilhelm} to dimension $3$  (at the cost of using a somewhat finer topology), see also \cite{Bryant} for a sketch of this result.


Theorem \ref{thm: main} implies that  building a convex hull from a given set  is always an infinite process in most Riemannian manifolds unlike the classical Euclidean case $\R^n$, where 
the convex hull is reached after at most $n$ steps.

\begin{thm}{Corollary}
 Let $g$ be a $\mathcal C^8$-generic complete Riemannian metric on any non-compact  manifold $M$ of dimension 
 at least $3$.  Let $A$ be any compact non-convex  subset of $M$, not contained in a minimizing  geodesic. Set $A=A_1$ and define  $A_k$ inductively as  the union of all minimizing geodesics connecting pairs of  points in 
$A_{k-1}$.  Then, for all $k$, the set $A_k$ is not  convex.
\end{thm}





The  $\mathcal C^8$-topology in the statement is definitely not optimal, at least for dimensions larger than $3$, see Theorem \ref{} below.  Note, that the genericity statements in all results above can be strenthend: in fact the set of all Riemannian metrics   
satisfying the conclusions of Theorem \ref{thm: main} and its Corollaries contains an open dense set with repsect to $\mathcal C^8$-topology.

     

 
 The proof is based on the following simple geometric observation, Proposition \ref{} below.
 Let a geodesic $\gamma $ in a Riemannian manifold $M$ be contained in the boundary of a convex 
 body $K\subset M$. Assume  that the tangent space $T_{\gamma (t)}K$ along $\gamma$ is a Euclidean halfspace. Then  the (in this case unique) normal vector $V(t)$ to $K$ at $\gamma (t)$
  is parallel along $\gamma$ and an eigenvector of the Jacobi operator along $\gamma$.
  
  This provides a non-trivial condition on the curvature tensor  and its derivative.  A final somewhat technical step is a transversality argument showing that generically such a condition is not fulfilled along no geodesic in a Riemannian manifold $M$.
  
  
\section{Jets of metric tenors}


\begin{thm}{Lemma}\label{lem:jacobi}
For any tangent vector $T$ at a point $p$ of a Riemannian manifold manifold $M$ and any integer $n$ there are three self-adjoint operators 
\[A_n=A_n(T),\ B_n=B_n(T),\ C_n=C_n(T)\:\T_p\to \T_p\]
that satisfy the following conditions:

\begin{subthm}{lem:jacobi:ABC}
For any Jacobi field $I$ along the geodesic $\gamma\:t\mapsto \exp_p(t\cdot T)$ the following identity holds at at $t=0$:
\[T^n\langle I,I\rangle
=
\langle A_n\cdot  I,I\rangle+ \langle B_n\cdot I,\nabla_T I\rangle+\langle C_n\cdot (\nabla_T I),\nabla_TI\rangle.\]
\end{subthm}

\begin{subthm}{lem:jacobi:R}
The functions $T\mapsto A_{n}(T)$, $T\mapsto B_{n+1}(T)$ and $T\mapsto C_{n+2}(T)$ are polynomial of degree $n$.
\end{subthm}

\begin{subthm}{lem:jacobi:A-R}
Set $R^{(k)}=R^{(k)}(T)\:V\mapsto (\nabla^k_T\Rm)(V,T)T$. Then the following differences  
\[A_{n+2}-2\cdot R^{(n)},\quad B_{n+3}-2\cdot(n+2)\cdot R^{(n)},\quad C_{n+4}-2\cdot(n+1)\cdot(n+4)\cdot R^{(n)}\]
depend only on $R,R',\dots, R^{(n-2)}$.
\end{subthm}


\end{thm}

\parit{Proof.}
Choose $T=T_0\in\T_p$;
extend it to a neighborhood of $p$ to a parallel vector field in the normal coordinates at $p$;
that is, set $T=(d_W\exp_p)T_0$ for small $W\in \T_p$.
Given a tensor field $S$, let us use the shortcut $S'$  for $\nabla_TS$.

Applying recursively the Jacobi identity to a Jacobi filed $I$ along the geodesic $\gamma\:t\mapsto\exp_p(t\cdot T)$, we get that the identity 
\[\langle I,I\rangle^{(n)}
=
\langle A_n\cdot  I,I\rangle+ \langle B_n\cdot I,I'\rangle+\langle C_n\cdot I',I'\rangle\]
holds at $t=0$,
where $A_0=1$, $B_0=0$, $C_0=0$, and the operators $A_n$, $B_n$, and $C_n$ satisfy the following recursive relations:
\begin{align*}
A_{n+1}&=A_n'-\tfrac12\cdot R\cdot B_n-\tfrac12\cdot B_n\cdot R
\\
B_{n+1}&=B_n'+ 2\cdot A_n-C_n\cdot R-R\cdot C_n
\\
C_{n+1}&=C_n'+ B_n
\end{align*}

It proves part \ref{SHORT.lem:jacobi:ABC};
by looking at the table, and applying induction, one gets~\ref{SHORT.lem:jacobi:R} and~\ref{SHORT.lem:jacobi:A-R}.\qeds

\renewcommand{\arraystretch}{1.5}
\begin{figure}[!ht]
\centering
\begin{tabular}{ l|c|c|c }
$n$ & $A$ & $B$ & $C$ \\ \hline
0& $1$   &  $0$  & $0$ \\ \hline
1& $0$   &  $2$  & $0$ \\ \hline 
2& $-2\cdot R$ & $0$ & $2$ \\ \hline 
3& $-2\cdot R'$ & $-8\cdot R$ & 0  \\ \hline 
4& $-2\cdot R''+8\cdot R^2$ & $-12\cdot R'$ & $-8\cdot R$  \\ \hline
5& $-2\cdot R^{(3)}+\dots$ 
& $-16\cdot R''+16\cdot R^2$ & $-20\cdot R'$  \\ \hline
6
&$-2\cdot R^{(4)}+\dots$
&$-20\cdot R^{(3)}+\dots$
&$-36\cdot R''+16\cdot R^2$
\\
\end{tabular}
\end{figure} 

Let $g$ be a Riemannian metric on a smooth manifold $M$.
Following \cite{eliashberg-mishachev}, we will denote by $J^n_g(p)$ its $n$-jet at the point $p$;
it cares information on all partial derivatives of $g$ at $p$ up to order $n$.

Assume $g$ is positive definite at $p$.
Note that $J^n_g(p)$ determines the covariant derivetives of the Riemannian curvature tensor $\nabla^k\Rm(p)$ at $p$ with order $k\z\le n-2$.
In other words, the map 
\[\rho_n\:J^n_g(p)\mapsto (\Rm(p),\dots,\nabla^{n-2}\Rm(p))\]
is defined.
Note that the domain $\Dom \rho_n\subset J^n_g(p)$ of $\rho_n$ is the open set of all $n$-jets such that $g$ is positive definite at $p$.

\begin{thm}{Proposition}
The map $\rho_n$ is a submersion;
that is for any jet $J^n_g(p)\in \Dom \rho_n$ the differential $d_{J^n_g(p)}\rho_n$ has a right inverse.
\end{thm}

\parit{Proof.}
Choose a Riemannian metric $g$ in a neighborhood of $p$.
Let $C_i=C_i(T)$ be as in \ref{lem:jacobi}.

Consider the normal coordinates at $p$.
Choose another vector $V=V_0\in \T_p$; extend it to a parallel vector field $V$ in the normal coordinates at $p$;
that is, $V=(d_W\exp_p)V_0$ for sufficiently small $W$.

Set $V_t=(d_{t\cdot T}\exp_p)V_0$ it is the restriction of $V$ to the geodesic $\gamma\:t\mapsto \exp_p(t\cdot T)$.
Set $I_t=t\cdot V_t$.
Note that $I$ is a Jacobi field along $\gamma$.
Observe that $t^2\cdot \langle V_t, V_t\rangle=\langle I_t, I_t\rangle$, $I(0)=0$, and $I'(0)=V(0)$.
Applying \ref{lem:jacobi}, we get that 
\[\langle V_t, V_t\rangle=\sum_{i=2}^{n+2}\tfrac1{(n+2)!}\cdot\langle C_n(T)\cdot V_0,V_0\rangle+o(t^n).\]
It follows that in the normal coordinates at $p$,
the $n$-jet $J^n_g(p)$ is uniquely described by the polynomials $T\mapsto C_2(T),\dots, T\mapsto C_{n+2}(T$.

Applying \ref{lem:jacobi} inductively on $n$, we get that the differential 
$d_{J^n_g(p)}\rho_n$ has a right inverse; hence the proposition follows.
\qeds 
 
\section{Notations}
 
A subset $K$ of a Riemannian manifold $M$ is called convex if  any pair of points $x,y \in K$
are connected by at least one minimizing geodesic $\gamma :[a,b]\to M$, which is  contained in $K$.


\section{Geometric arguments}
\subsection{Structure of convex subsets}
Let $K$ be a closed convex subset of an $n$-dimensional  Riemannian manifold $M$. Then, for some natural $k\leq n$, the subset $K$ is a topological manifold with boundary $\partial K$.

If $k=n$ then $\partial K$ coincides with the topological boundary of $K$ as subset of $M$ and 
$K\setminus \partial K$ is the set of inner points of $K$ in $M$.

If $k<n$ then $K\setminus \partial K$ is a totally geodesic submanifold of $K$.





\subsection{Geodesic in the boundary}
Let $M$ be  smooth Riemannian manifold, let $K\subset M$


\subsection{Jacobi fields}
Let $C$ be a closed cone with non-empty interior $C^0$ in the Euclidean space $\R^m$. Let $R(t)$ be a smooth family of symmetric endomorphisms of $\R^m$ and let $\mathcal J$ denote the vector space of \emph{Jacobi fields} 
$J:(-\epsilon, \epsilon) \to \R^m$  governed by the \emph{Jacobi equation} $J''(t)+R(t) \cdot J(t) =0$.

Assume that the following condition holds true:  For any $J\in \mathcal J$, and any $a<m<b$ in the interval of definition of $R$, the inclusions $J(a),J(b) \in C$ implies that $J(m)\in C$.

We claim:
\begin{thm}{Lemma}
	Under the above assumption, for any $t$ in the interval of definition $I$, we have 
	$-R(t) (\partial C)  \cap C^0 =\emptyset$. 
\end{thm}

\parit{Proof.}
	Assume the contrary. W.l.o.g. assume that the condition is violated at $t=0$.
	Thus, for $R=R(0)$ and some $v\in \partial C$ we have $-R(v)\in C^0$.
	
	Consider the Jacobi field $J(t)$ uniquely defined by 
	$J(0)=v$ and $J'(0)=0$.  
	Then, for all small $t$ we have:
	$$J(t)= v  -  t ^2 \cdot  (R (v)) + O(t^3)\;.$$
	By assumption,  $J(t)$ is contained in the interior $C^0$ for small $t \neq 0$.
	
	Take $\delta$ so small that on the interval $(-2\delta, 2\delta)$ the Jacobi equation has no conjugate points.
	Then, for the point $w^-= J(-\delta)$, and any point  $w$ in a neighborhood of $w^+ =J(\delta)$ there exists exactly one Jacobi field $J^w$
		with $J(-\delta)=w^-$ and $J^w(\delta )= w$.  Moreover, the map $w\to J^w(0)$ is a local diffeomorphism.
		
		Thus, we find some $w$ close to $w^+$, such that $J^w (\pm \delta) \in C^0 \subset C$ and $J^w(0)$ is not contained in $C$. 
\qeds


\subsection{Curvature conditon}



\section{Transversality}
\subsection{Jets of the metric and of the curvature tensor}




\subsection{Linear algebra}
We begin with the following simple:
\begin{thm}{Lemma}
	Let $k,m\geq 2$ be a natural number and let  $S=Sym _m$ be the vector space of symmetric 
	$m\times m$ matrices.  Let $E_{m,k} \subset S^k$ denote   the set of all $k$-tuples $(A_1,...A_k) \in S^k$  having a common eigenvector.   Then $E_{m,k}$ is an algebraic subset
	of $S^k$ of codimension  $(m-1)\cdot (k-1)$.
\end{thm}






\bibliographystyle{alpha}
\bibliography{convex}
\end{document}
