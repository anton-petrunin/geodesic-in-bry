\documentclass[a4paper,10pt]{article}
\usepackage{paper}

\def\thetitle{About all convex sets in\\ most Riemannian manifolds}
\hypersetup{
pdftitle={\thetitle},
pdfauthor={Alexander Lytchak and Anton Petrunin}
}

\begin{document}

\title{\thetitle}
\author{Alexander Lytchak and Anton Petrunin}
\date{}
\maketitle

\begin{abstract}
We give a necessary condition on a geodesic in a Riemannian manifold that can run in some convex hypersurface.
As a corollary we obtain peculiar properties of \emph{all} convex sets that hold true in any \emph{generic} Riemannian manifold $(M,g)$.
For example, if a convex set in $(M,g)$ is bounded by a smooth hypersurface, then it is strictly convex.
\end{abstract}

\section{Introduction}
Let $\mathfrak{C}$ be the convex hull of a subset $Q$ in a Euclidean space.
By Carathéodory's theorem, any point in $\mathfrak{C}\setminus Q$ is an inner point of a line segment contained in $\mathfrak{C}$;
that is, the complement $\mathfrak{C} \setminus Q$ does not contain extreme points of $\mathfrak{C}$.
This statement admits a straightforward generalization to the sphere and the Lobachevskian space.

This statement is also easily seen to hold \emph{locally} in any two-dimensional Riemannian manifold.
(Recall that a set $\mathfrak{C}$ in a Riemannian manifold $(M,g)$ is called \emph{convex} if for any pair of points $x,y\in \mathfrak{C}$ any minimizing geodesic $[x,y]$ lies in $\mathfrak{C}$.
A point in $\mathfrak{C}$ is called \emph{extreme} if it does not lie in an interior of a geodesic in~$\mathfrak{C}$.)

It seems to be a folklore belief that a version of this statement should hold true in all Riemannian manifolds.
For instance, it is claimed by Mohammad Ghomi and Joel Spruck \cite[Lemma 9.1]{Ghomi}.
In the present note we prove that the somewhat counter-intuitive opposite is the case for \emph{generic} Riemannian manifolds.
It agrees with the pattern: \emph{a random object in your favorite theory  looks like nothing you have ever seen before}.

Further Riemannian metrics will be assumed to be $\mathcal C^\infty$-smooth.
Given a positive integer $k$, we say that a property $\mathcal P$ holds for \emph{$\mathcal C^k$-generic} Riemannian metric $g$ on a manifold $M$ 
if the property $\mathcal P$ holds for a dense G-delta set of metric tenors in the $\mathcal C^k$-topology.
Since the space of Riemannian metrics with $\mathcal C^k$-topology on $M$ is \emph{Baire}, 
countable intersection of dense G-delta sets, is dense.
Therefore one can say that \emph{most} Riemannian metrics are generic.

\begin{thm}{Main theorem}\label{thm:main}
Let $g$ be a $\mathcal C^2$-generic Riemannian metric on a smooth manifold $M$.
Then any closed convex set $\mathfrak{C}\subset (M,g)$ that is not a subset of a geodesic
satisfies the following properties:

\begin{subthm}{thm:main:interior}
The set $\mathfrak{C}$ has nonempty interior.
\end{subthm}

\begin{subthm}{thm:main:geodesics}
The set of all non-extreme points $\partial\mathfrak{C}$ is a countable union of disjoint geodesics in $(M,g)$.
In particular, if $\dim M\ge 3$, then the set of
extreme points of $\mathfrak{C}$ is dense in~$\partial\mathfrak{C}$.
\end{subthm}

\end{thm}

Part \ref{SHORT.thm:main:interior} for $\dim M\ge 4$ is proved by Thomas Murphy and Frederick Wilhelm \cite{Wilhelm};
for $\dim M=3$ the proof was sketched by Robert Bryant \cite{Bryant}.

Recall that convex hull of a set $Q$ is defined as the minimal convex set (with respect to inclusion) that contains $Q$.
Note that according to Carathéodory theorem, any compact set of a Euclidean space has a compact convex hull.
According to the following corollary, the latter does not hold in most Riemannian manifolds.

\begin{thm}{Corollary}\label{cor:caratheodory}
Let $g$ be a $\mathcal C^2$-generic Riemannian metric on a smooth connected manifold $M$ with dimension at least 3.
Suppose a connected set $Q$ has compact convex hull $\mathfrak{C}$.
Then either $Q\supset \partial \mathfrak{C}$ or $Q$ is contained in a minimizing geodesic of $(M,g)$.
\end{thm}

The proofs are built on the following proposition.
Its formulation use the notion of \emph{rank} of convex set at a point which is defined as the dimension of maximal linear subspace in the tangent cone to the convex set at the point;
a formal definition is given in the following section.

\begin{thm}{Main proposition}\label{prom:rank}
Let $g$ be a $\mathcal C^2$-generic Riemannian metric on a smooth $m$-dimensional manifold $M$.
Suppose $\mathfrak{C}$ is a closed convex set in $(M,g)$.
Then all non-extreme points of $\mathfrak{C}$ have rank either $1$ or $m$.

In particular, if $\dim M\ge 3$ and $\mathfrak{C}$ is bounded by a smooth hypersurface, then it is \emph{strictly convex}; that is, all boundary points of $\mathfrak{C}$ are extreme.
\end{thm}

A proof of a weaker version of the proposition was sketched by the second author~\cite{petrunin-2009}.

The proof of the proposition is built on the key lemma stated in the following section; 
it describes a necessary condition on a geodesic in a Riemannian manifold that can run in some convex, possibly non-smooth, hypersurface.
If the geodesic contains a point of rank at least 2, then this condition implies a non-trivial
statement on the curvature tensor of the manifold.
The remaining part is straightforward: we show that the curvature tensor of a generic Riemannian manifold does not meet this property; this is done by applying the Thom transversality theorem.


\section{Key lemma}


Let $\mathfrak{C}$ be a closed convex set in an $m$-dimensional Riemannian manifold $(M,g)$.
Recall that $\T_x=\T_xM$ denotes the \emph{tangent space} of $M$ at $x$.
The \emph{tangent cone} $\K_x\mathfrak{C}\subset \T_x$ at $x\in\mathfrak{C}$ is defined as the closure of the set of all velocity vectors of geodesics that start at $x$ and run in $\mathfrak{C}$.

Given $x\in \mathfrak{C}$, denote by $\L_x\mathfrak{C}$ the \emph{maximal linear subspace} of $\K_x\mathfrak{C}$.
The \emph{rank} of $\mathfrak{C}$ at $x$ is defined as $\dim \L_x\mathfrak{C}$.

Note that the $\K_x\mathfrak{C}$ is a convex cone in $\T_x$; in particular, $\L_x=\K_x\cap (-\K_x)$.
Further $\K_x\mathfrak{C}$ coincides with 
$\T_x$ if and only if
$x$ lies in the interior of $\mathfrak{C}$.
In other words, $x$ has rank $m$ if and only if $x$ lies in the interior of $\mathfrak{C}$.%???REF, see, for instance, \cite{}.


Given a tangent vector $\vec{x}\in\T_pM$, consider the so called \emph{Jacobi operator}
\[R(\vec{x})\:\vec{v}\mapsto \Rm(\vec{v},\vec{x})\vec{x},\]
where $\Rm$ denotes the curvature tensor of $g$.
Note that $R(\vec{x})\:\T_p\to \T_p$ is a self-adjoint operator.
Note that the Jacobi equation along a geodesic $\gamma$ takes form 
\[\vec{i}''+R(\gamma')\cdot \vec{i}=0,\]
here $\vec{i}''$ is a shortcut for $\nabla^2_{\gamma'}\vec{i}$.

\begin{thm}{Key lemma}\label{lem:key}
Let $(M,g)$ be a Riemannian manifold and $\gamma\:(a_0,b_0)\to M$ be a geodesic that runs in a closed convex set $\mathfrak{C}\subset (M,g)$.
Then the tangent cone of $\mathfrak{C}$ is parallel along $\gamma$; that is, the parallel translation along $\gamma$ defines a bijection between the tangent cones $\K_{\gamma(a)}\mathfrak{C}$ and $\K_{\gamma(b)}\mathfrak{C}$ for any $a,b \in (a_0,b_0)$.

Moreover, for any $a\in (a_0,b_0)$ the following hold:
\begin{subthm}{lem:key:a} %???Maybe make (1) and (2) instead of (a) and (b)? Or change the notation of $a$ and $b$ for times on $\gamma$
For any $\vec{v}\in \K_{\gamma(a)}\mathfrak{C}$ we have
\[R(\gamma'(a))\cdot \vec{v}\in \K_\vec{v}[\K_{\gamma(a)}\mathfrak{C}].\]
\end{subthm}

\begin{subthm}{lem:key:b} 
$\L_{\gamma(a)}\mathfrak{C}$ is an invariant subspace of $R(\gamma'(a))$.
\end{subthm}

\end{thm}

The proof uses the fact that the parallel translation can be defined via geodesics.
In a similar way, this observation was used in \cite[Section 13]{Ber-Nik} and~\cite{Petruninpar}.
In fact the main part of the key lemma follows from the result in \cite{Petruninpar}.

\parit{Proof of \ref{lem:key}.} 
Since all the statements are local, we may replace $(M,g)$ by its small open convex subset.
By doing so we may assume that any pair of points of $(M,g)$ are connected by a unique geodesic and it has no conjugate points.
In particular, for any subinterval $[a,b]\subset (a_0,b_0)$ and any tangent vectors $\vec{v} \in \T_{\gamma (a)}$ and $\vec{w} \in \T_{\gamma (b)}$ there exists unique Jacobi field $\vec{i}$ along $\gamma$ such that $\vec{i}(a)=\vec{v}$ and $\vec{i}(b)=\vec{w}$.

Since Jacobi fields are variational fields of geodesic variations, the convexity of $\mathfrak{C}$ implies the following: 

\begin{thm}{Observation}
%	Let $\mathfrak{C}$ be a closed convex subset in a Riemannian manifold $(M,g)$.
Suppose $\vec{i}$ is a Jacobi field along % a geodesic
$\gamma$ and $a_0<a<t<b<b_0$.
If 
% \:[a,b]\to \mathfrak{C}$ in $(M,g)$.
%Assume that
$\vec{i}(a)\in \K_{\gamma(a)}\mathfrak{C}$ and $\vec{i}(b)\in \K_{\gamma(b)}\mathfrak{C}$
then $\vec{i}(t)\in \K_{\gamma(t)}\mathfrak{C}$.
	% for any $t\in [a,b]$.
\end{thm}%???Это наблюдение стало менее общим, изза этого его (возможно) будет тяжелее понимать???

Choose a subinterval $[a,b] \subset (a_0,b_0)$.
Given a large positive integer $k$, consider arithmetic progression
$t_0,\dots,t_{k+1}$ such that $t_0=a$ and $t_k=b$.

Choose a tangent vector $\vec{v}_0\in\T_{\gamma(a)}$.
Consider the sequence of vectors $\vec{v}_i\z\in\T_{\gamma(t_i)}$ defined recursively by $\vec{v}_{i+1}=2\cdot \vec{i}_i(t_{i+1})$, where $t\mapsto \vec{i}_i(t)$ is the Jacobi field along $\gamma$ such that $\vec{i}_i(t_i)=\vec{v}_i$ and $\vec{i}_i(t_{i+2})=0$.

Define $\iota_k\:\T_{\gamma(a)}\to \T_{\gamma(b)}$ by setting $\iota_k(\vec{i}_0)\df \vec{i}_k$.
According to the observation, if $\vec{i}_0\in \K_{\gamma(a)}\mathfrak{C}$, then $\iota_k(\vec{i}_0)\in \K_{\gamma(b)}\mathfrak{C}$.
Note that $\iota_k(\vec{i}_0)$ converges to the parallel translation of $\vec{i}_0$ along $\gamma$ as $k\to \infty$.
Since $\K_{\gamma(b)}\mathfrak{C}$ is closed,
the parallel translation along $\gamma$ maps $\K_{\gamma(a)}\mathfrak{C}$ in $\K_{\gamma(b)}\mathfrak{C}$.
Switching the direction of $\gamma$, we get the opposite inclusion.
That is, the tangent cone $\K_{\gamma(t)}\mathfrak{C}$ is parallel along $\gamma$ --- the main part is proved.

Let us use parallel translation along $\gamma$ to identify the tangent spaces at the points on $\gamma$.
This way we will identify the tangent cones $\K_{\gamma(t)}\mathfrak{C}$ for all $t$;
denote the obtained cone by $\K$.

For $\vec{v}\in \K$ and small $\epsilon>0$ consider the unique Jacobi field $\vec{i}_\epsilon$ along $\gamma$ with $\vec{i}_\epsilon (a+\eps)\z=\vec{i} _\epsilon(a-\eps)=\vec{v}$.
Due to the Jacobi equation,
\[\vec{i}_\epsilon (a)=\vec{v} +\eps^2\cdot R(\gamma'(a))\cdot \vec{v} +o(\eps^2).\]
According to the observation, $\vec{i}_\epsilon(a)\in \K$ for any $\eps>0$.
Since $\K$ is a closed convex cone, we get $R(\gamma')\cdot \vec{v}\z\in \K_\vec{v}\K$ --- part \ref{SHORT.lem:key:a} is proved.

Finally note that 
\[\vec{v}\in \L_{\gamma(a)}\mathfrak{C}
\quad\iff\quad 
\vec{v}, -\vec{v}\in \K
\quad\iff\quad 
\K_\vec{v}\K=\K.
\]
Therefore, if $\vec{v}\in \L_{\gamma(a)}\mathfrak{C}$, then $\pm R(\gamma'(a))\cdot \vec{v}\in \K$ and hence $R(\gamma'(a))\cdot \vec{v}\in \L_{\gamma(a)}\mathfrak{C}$.
That is, $\L_{\gamma(a)}\mathfrak{C}$ is an invariant subspace of $R(\gamma'(a))$ --- part \ref{SHORT.lem:key:b} is proved.
\qeds

\section{Taylor expansion of metric tensor}\label{sec:jet}

In following lemma makes possible to calculate the Taylor expansion of metric tensor in the normal coordinates in terms of curvature tenor and its covariant derivatives.
This formula was derived by Old\v{r}ich Kowalski and Martin Belger \cite[Proposition 2.2]{kowalski-belger} (that is the earliest reference we have found).
We present calculations since some notations will be used in the proof of \ref{prop:submersion}.

Let $(M,g)$ be a Riemannian manifold.
Recall that $\Rm$ denotes the curvature tensor.

Choose a tangent vector $\vec{x}_p\in \T_pM$; 
extend it to a parallel vector field $\vec{x}$ in the normal coordinates at $p$.
That is, set 
\[\vec{x}_{\exp_p\vec{w}}=(d_\vec{w}\exp_p)\vec{x}_p\] for all small $\vec{w}\in \T_p$.
Note that $(\nabla^k_\vec{x} \vec{x})_p=0$ for any $k\ge 1$.

Given a tensor field $S$, let us use the shortcut $S'$ for $\nabla_\vec{x}S$ at $p$.
Recall that $R=R(\vec{x})$ denotes the Jacobi operator of $\vec{x}$.
The \emph{higher Jacobi operators} $R^{(k)}$ are defined by
\[R^{(k)}=R^{(k)}(\vec{x})\df\nabla^k_\vec{x}R(\vec{x}).\]
Note that 
\begin{itemize}
\item $R^{(k)}(\vec{x})\:\T_p\to\T_p$ is a self-adjoint operator, 
\item $\vec{x}\perp R^{(k)}(\vec{x})\cdot \vec{v}$ for any vector field $\vec{v}$, and
\item $\vec{x}\mapsto R^{(k)}(\vec{x})$ is a homogenius polynomial of degree $k+2$ defined for all $\vec{x}\in \T_p$.
\end{itemize}




\begin{thm}{Lemma}\label{lem:jacobi}
For any tangent vector $\vec{x}$ at a point $p$ of a Riemannian manifold manifold $(M,g)$ and any integer $n$ there are three self-adjoint operators 
\[A_n=A_n(\vec{x}),\ B_n=B_n(\vec{x}),\ C_n=C_n(\vec{x})\:\T_p\to \T_p\]
that satisfy the following conditions:

\begin{subthm}{lem:jacobi:ABC}
For any Jacobi field $\vec{i}$ along the geodesic $\gamma\:t\mapsto \exp_p(t\cdot \vec{x})$ the following identity holds:
\[\langle \vec{i},\vec{i}\rangle^{(n)}
=
\langle A_n\cdot \vec{i},\vec{i}\rangle+ \langle B_n\cdot \vec{i}, \vec{i}'\rangle+\langle C_n\cdot \vec{i}', \vec{i}'\rangle.\]
\end{subthm}

\begin{subthm}{lem:jacobi:R}
$\vec{x}\mapsto A_{n}(\vec{x})$, $\vec{x}\mapsto B_{n+1}(\vec{x})$ and $\vec{x}\mapsto C_{n+2}(\vec{x})$ are homogeneous polynomials of degree $n$ defined for all $\vec{x}\in \T_p$.
\end{subthm}

\begin{subthm}{lem:jacobi:A-R}
The following differences 
\[A_{n+2}-2\cdot R^{(n)},
\quad 
B_{n+3}-4\cdot(n+2)\cdot R^{(n)},
\quad
C_{n+4}-2\cdot(n+1)\cdot(n+4)\cdot R^{(n)}\]
depend only on $R,R',\dots, R^{(n-2)}$.
\end{subthm}


\end{thm}

\parit{Proof.}
Applying recursively the Jacobi identity $\vec{i}''=-R\cdot \vec{i}$ for the geodesic $\gamma\:t\mapsto\exp_p(t\cdot \vec{x})$, we get that the following identity 
\[\langle \vec{i},\vec{i}\rangle^{(n)}
=
\langle A_n\cdot \vec{i},\vec{i}\rangle+ \langle B_n\cdot \vec{i},\vec{i}'\rangle+\langle C_n\cdot \vec{i}',\vec{i}'\rangle\]
where $A_0=1$, $B_0=0$, $C_0=0$, and the operators $A_n$, $B_n$, and $C_n$ satisfy the following recursive relations:
\begin{align*}
A_{n+1}&=A_n'-\tfrac12\cdot R\cdot B_n-\tfrac12\cdot B_n\cdot R
\\
B_{n+1}&=B_n'+ 2\cdot A_n-C_n\cdot R-R\cdot C_n
\\
C_{n+1}&=C_n'+ B_n
\end{align*}

It proves part \ref{SHORT.lem:jacobi:ABC};
by looking at the table, and applying induction, one gets~\ref{SHORT.lem:jacobi:R} and~\ref{SHORT.lem:jacobi:A-R}.\qeds

\renewcommand{\arraystretch}{1.5}
\begin{figure}[!ht]
\centering
\begin{tabular}{ l|c|c|c }
$n$ & $A$ & $B$ & $C$ \\ \hline
0& $1$ & $0$ & $0$ \\ \hline
1& $0$ & $2$ & $0$ \\ \hline 
2& $-2\cdot R$ & $0$ & $2$ \\ \hline 
3& $-2\cdot R'$ & $-8\cdot R$ & 0 \\ \hline 
4& $-2\cdot R''+8\cdot R^2$ & $-12\cdot R'$ & $-8\cdot R$ \\ \hline
5& $-2\cdot R^{(3)}+\dots$ 
& $-16\cdot R''+16\cdot R^2$ & $-20\cdot R'$ \\ \hline
6
&$-2\cdot R^{(4)}+\dots$
&$-20\cdot R^{(3)}+\dots$
&$-36\cdot R''+16\cdot R^2$
\\
\end{tabular}
\end{figure} 

Following \cite{eliashberg-mishachev}, we will denote by $J^n_g(p)$ the $n$-jet of a metric tensor $g$ at the point $p$;
it cares information on all partial derivatives of $g$ at $p$ up to order~$n$ in some (and therefore every) local coordinate system at $p$.
Equivalently, the jet $J^n_g(p)$ can be described by Taylor expansion up to degree $n$ of $g$ at $p$ (in some, and therefore every, local coordinate system at $p$).
Let us denote by $\mathcal{J}^n(p)$ the space of all $n$-jet of a Riemannian metric tensors at the point $p$

Assume $g$ is positive definite at $p$.
Note that $J^n_g(p)$ determines the polynomial functions 
\[\vec{x}\mapsto R(\vec{x}),
\quad \vec{x}\mapsto R'(\vec{x}),\quad\dots,\quad \vec{x}\mapsto R^{(n-2)}(\vec{x})\eqlbl{eq:R_(n-2)}\] 
defined for all $\vec{x}\in \T_p$.
In other words, the map 
\[\rho_n\:J^n_g(p)\mapsto (R,\dots,R^{(n-2)})\]
is defined.

The image of $\rho_n$; that is,
the space of all possible polynomials in \ref{eq:R_(n-2)} will be denoted by $\mathcal{R}^{n-2}(p)$.
By \cite[Theorem 1.1]{kowalski-belger}, the space $\mathcal{R}^{n-2}$ is a linear with respect to standard addition and multiplication by a scalar.

\begin{thm}{Proposition}\label{prop:submersion}
For any jet $J^n_g(p)\z\in \mathcal{J}^n(p)$ the differential $d_{J^n_g(p)}\rho_n$ has a right inverse $s$.
Moreover $s$ can be chosen so that the jets in its image correspond to metric tensors that share the normal coordinates with $g$.

In particular, the map $\rho_n\:\mathcal{J}^n(p)\to \mathcal{R}^{n-2}(p)$ is a submersion.
\end{thm}

\parit{Proof.}
Choose a Riemannian metric $g$ in a neighborhood of $p$.
Let $C_i=C_i(\vec{x})$ be as in \ref{lem:jacobi}.

Choose a vector filed $\vec{v}$ that is parallel in the normal coordinates at~$p$;
that is, $\vec{v}_{\exp_p\vec{w}}=(d_\vec{w}\exp_p)\vec{v}_p$ for some $\vec{v}_p\in\T_p$ and all small $\vec{w}\in \T_p$.

Set $\vec{v}(t)=\vec{v}_{\exp_p(t\cdot \vec{x})}$ and $\vec{i}(t)=t\cdot \vec{v}(t)$.
Note that $\vec{i}$ is a Jacobi field along the geodesic $t\mapsto \exp_p(t\cdot \vec{x})$.
Observe that 
\[t^2\cdot \langle \vec{v}(t),
\vec{v}(t)\rangle=\langle \vec{i}(t), \vec{i}(t)\rangle,
\quad 
\vec{i}(0)=0,
\quad 
\text{and} \quad 
\vec{i}'(0)\z=\vec{v}_p.\]
Applying \ref{lem:jacobi}, we get
\[\langle\vec{i}(t), \vec{i}(t)\rangle=\sum_{i=0}^{n+2}\tfrac1{(n+2)!}\cdot\langle C_n(\vec{x})\cdot \vec{v}_p,\vec{v}_p\rangle\cdot t^{n+2}+o(t^{n+2}).\]
Applying the last identity for vector $\vec{x}/|\vec{x}|$ and $t=|\vec{x}|$, we get the equality
\[\langle\vec{v}, \vec{v}\rangle=\sum_{i=2}^{n+2}\tfrac1{(n+2)!}\cdot\langle C_n(\vec{x})\cdot \vec{v}_p,\vec{v}_p\rangle+o(|\vec{x}|^{n})\]
at the point $\exp_p \vec{x}$.
It follows that in the normal coordinates at $p$,
the $n$-jet $J^n_g(p)$ is uniquely described by the polynomial functions
\[\vec{x}\mapsto C_2(\vec{x}),
\ \dots\ ,\ 
\vec{x}\mapsto C_{n+2}(\vec{x}).\]

It remains to apply \ref{lem:jacobi:A-R} inductively on $n$.
\qeds 

\section{Genericity }

Recall that $R^{(k)}(\vec{x})\cdot \vec{x}=0$
for any $k$ and $\vec{x}\in \T_p$.
Therefore the orthogonal complement $\vec{x}^{\perp}$ of $\vec{x}$ as well as the subspace spanned by $\vec{x}$ are invariant subspace of $R^{(k)}(\vec{x})$ for any $k$.

Therefore for any integer $k\ge 0$,
the operator $R^{(k)}(\vec{x})$ has 4 \emph{trivial} invariant subspaces of dimension $0$, $1$ (the subspace spanned by $\vec{x}$), $n-1$ (the orthogonal complement to $\vec{x}$) and $m=\dim M$;
the remaining invariant subspaces will be regarded as \emph{nontrivial}.

Suppose a nonconstant geodesic $\gamma$ in a Riemannian manifold has a parallel family of subspace $V_t \subset \T_{\gamma (t)}$ that are nontrivial invariant subspaces of $R(\gamma'(t))$ for any $t$.
In this case we say briefly that $\gamma$ admits an \emph{invariant family of subspaces}.

Let $M$ be an $m$-dimensional smooth manifold.
Choose a compact subset $K\subset M$ and $\eps>0$.
Consider the set $G(K,\eps)$ of all Riemannian metrics $g$ on $M$ such that there exists a geodesic $\gamma$ in $(M,g)$ that starts in $K$, has length $\eps$, and admits an invariant family of subspaces.

Since the geodesics and the curvature tensor depend continuously on the Riemannian metric in $\mathcal C^2$-topology, we get that 

\begin{thm}{Claim}\label{clm:G}
The set $G(K,\eps)$ is closed with respect to the $\mathcal C^2$-topology on the space of all Riemannian metrics on $M$.
\end{thm}

Denote by $G^k(K)$ the set of all smooth Riemannian metrics, such that for some $p\in K$ and a non-zero tangent vector $\vec{x}\in\T_p$ all the operators $R(\vec{x}),R'(\vec{x}),\dots,R^{(k)}(\vec{x})$ have a common nontrivial invariant subspace $V_0\subset \T_p$.

By the very definition of $G^k(K)$ we get the following:

\begin{thm}{Claim} $G^k(K)$ is closed with respect to $\mathcal C^{k+2}$-topology on the space of all Riemannian metrics on $M$.
\end{thm}

Suppose that $g\in G(K,\eps)$ and let $\gamma$ is a geodesic as above starting at $p\in K$ in the direction $\vec{x}$.
Taking covariant derivatives along $\gamma$, we get the following:

\begin{thm}{Claim}
$G^k(K)\z\supset G(K,\eps)$ for any integer $k\ge 0$, compact set $K\subset M$, and $\eps>0$.
\end{thm}


\begin{thm}{Proposition}\label{prop:G}
Let $M$ be a smooth manifold of dimension $m\ge3$.
Then, for any $k\geq 6$, the set $G^k(K)$ is nowhere dense in the space of all Riemannian metrics.
\end{thm}

\parit{Proof.}
Choose nonzero $\vec{x}\in \T_p$.
Note that $k$-tuples $(R(\vec{x}),...,R^{(k)}(\vec{x}))$ that have a common nontrivial invariant subspace is an algebraic subset of~$\mathcal{R}^k(p)$.
Observe that the codimension of this subset is $(m-2)\z\cdot (k-1)$ in $\mathcal{R}^k(p)$.
The latter follows since the self-adjoint operators $R(\vec{x}),...,R^{(k)}(\vec{x})$ can be arbitrary, provided $R^{(i)}(\vec{x})=0$ for any $i$.

Consider the union $\Sigma$ of these subspaces for all unit vectors $\vec{x}\in \T_p$.
Note that $\Sigma\subset \mathcal{R}^k(p)$ is algebraic and its codimension is at least 
\[\ell=(m-2)\z\cdot (k-1)-(m-1).\]

By \ref{prop:submersion}, $\rho^{-1}(\Sigma)$ has codimension at least $\ell$ in the space of jets $\mathcal{J}^{k+2}(p)$.

Since $m\ge 3$, we get that $\ell> m$ if $k\ge 6$.
It remains to apply the Thom transversality theorem \cite[2.3.2]{eliashberg-mishachev}.
\qeds

Let $M$ be a smooth manifold.
Choose a nested sequence of compact sets $K_1\z\subset K_2\subset \dots$ that cover whole $M$ and set $\eps_n=\tfrac1n$.
Set 
\[G(M)=\bigcup_n G(K_n,\eps_n).\]
Note that $g\in G(M)$ if and only if $(M,g)$ has a nonconstant geodesic with an invariant family of subspaces.

Note that the statements in this section imply the following:

\begin{thm}{Claim}\label{clm:meager}
The set $G(M)$ is $\mathcal{C}^2$-meager; that is, its complement is a dense G-delta set in the space of all Riemannian metrics on $M$ with $\mathcal{C}^2$-topology.

In other words, if $g$ is a $\mathcal C^2$-generic Riemannian metric on a smooth manifold $M$,
then $(M,g)$ does not have nonconstant geodesic with an invariant family of subspaces.
\end{thm}


\section{Structure of convex sets}

The following claim makes it possible to extend some results about convex sets in Euclidean space to Riemannian manifolds.

\begin{thm}{Claim}\label{clm:convex}
Suppose that $\mathfrak{C}$ is a closed convex set in a Riemannian manifold $(M,g)$ and $x\in \mathfrak{C}$.
Then there is a smooth chart $s\:U\to M$ that covers $x$ such that the inverse image $s^{-1}\mathfrak{C}$ is a convex subset in $U$.
\end{thm}

\parit{Proof.}
Choose normal coordinates $(u_1,\dots, u_m)$ in a small spherical neighborhood of $x$.
We may assume that the $u_1$-axis point in the relative interior of~$\K_x\mathfrak{C}$.

Set $w_1=u_1-u_2^2-\dots-u_m^2$ and observe that the coordinates
$(w_1,u_2,\dots, u_m)$ do the trick.
\qeds

Since any closed convex set in Euclidean space is a topological manifold with boundary, applying \ref{clm:convex} we get the following:

\begin{thm}{Proposition}\label{prop:mnfld}
Any connected closed convex set $\mathfrak{C}$ in a Riemannian manifold $(M,g)$ is homeomorphic to a manifold with boundary, say $\mathfrak{B}$.
Moreover, the complement $\mathfrak{C}\backslash \mathfrak{B}$ is a totaly geodesic submanifold of $(M,g)$.
\end{thm}

Applying the result of Luděk 
Zajíček \cite{zajicek} 
together with \ref{clm:convex} we also get the following:

\begin{thm}{Proposition}\label{prop:rectifiable}
Let $\mathfrak{C}$ be a closed convex set in a Riemannian manifold $(M,g)$.
Then the set of points in $\mathfrak{C}$ with rank at most $k$ is countably \emph{$k$-rectifiable};
that is, this set can be  covered by images of a countable set of Lipschitz maps $\RR^k\to (M,g)$.
\end{thm}

\section{Proof assembling}

\parit{Proof of the main proposition (\ref{prom:rank}).}
Apply \ref{clm:meager} and the key lemma \mbox{(\ref{lem:key:b})}.


\parit{Proof of the main theorem (\ref{thm:main}).}
Let $\mathfrak B$ be as in \ref{prop:mnfld}; recall that $\mathfrak C\setminus \mathfrak B$ is a geodesic submanifold of $(M,g)$;
denote its dimension by $d$.
The tangent cone $\K_p \mathfrak C$ at any $p \in \mathfrak C\setminus \mathfrak B$ is $d$-dimensional linear space.
By Proposition \ref{prom:rank},, $d=1$ or $d=m$.
In the first case, $\mathfrak C\setminus \mathfrak B$ is a geodesic in $(M,g)$;
hence $\mathfrak C$ is contained in a geodesic as well.
In the latter case, $\mathfrak C\setminus \mathfrak B$ is open in $M$ --- part \ref{SHORT.thm:main:interior} is proved.

By Proposition \ref{prom:rank} any non-extreme point $x\in \partial \mathfrak C$ has rank 1.
Thus there is unique line in $\K_x\mathfrak C$ and it is the tangent line of the (therefore unique) geodesic $\gamma$ contained in $\mathfrak C$ and having $p$ as an inner point. 

Let us extend $\gamma$ to a maximal open interval, so that $\gamma$ stays in $\mathfrak C$.
By the main statement of the key lemma, all points on $\gamma$ lie on $\partial \mathfrak C$.
By the definition, all such geodesics consist of non-extreme points.

As seen above, through each non-extreme point on $\mathfrak C$ passes exactly one such geodesics.
Thus we have subdivided the set of non-extreme points of $\partial\mathfrak C$ into $g$-geodesics with positive lengths.

Finally, by \ref{prop:rectifiable}, there only countably many of such geodesics.
\qeds

\parit{Proof of the corollary (\ref{cor:caratheodory}).}
The convex hull $\mathfrak{C}$ of $Q$ can be obtained as the following infinite union.
We start with $Q=Q_0$ and define inductively $Q_{i+1}$ to be the union of all minimizing geodesics between pairs of points of $Q_i$.
Then the union $\mathfrak{C}= \bigcup_{i} 
Q_i$ is the convex hull of $Q$.

Assume that $\mathfrak{C}$ is closed and not equal to $M$; in particular $\partial\mathfrak{C}\ne \varnothing$.
Since $Q$ is not contained in a geodesic, by the main theorem, $\mathfrak{C}$ has a non-trivial interior.
By the construction of $\mathfrak{C}$ above, any point $x\in \mathfrak{C} \setminus Q$ is not an extreme point of~$\mathfrak{C}$.

Assume $\partial \mathfrak{C} \not\subset Q$.
By the main theorem the topological manifold $\partial \mathfrak{C}$ is the union of the compact subset $Q\cap \partial \mathfrak{C}$ and a countable union of geodesics.
But $\partial \mathfrak{C} \setminus Q$ is an $(m-1)$-dimensional topological manifold.
In particular it is not a union of countably many disjoint simple curves --- a contradiction.
\qeds

\section{Final remarks}

Recall that according to the main proposition (\ref{prom:rank}), if a geodesic $\gamma$ of a generic Riemannian manifold $(M,g)$ runs in the boundary of convex set $\mathfrak{C}$, then any point on $\gamma$ has rank 1.

It worth to mention the following closely related result of Anatoliy Milka \cite[§~4]{milka} in the opposite direction; originally it was proved for Euclidean space, but the same proof works for Riemannian manifolds:

\begin{thm}{Milka's theorem}
Let $\mathfrak{C}$ be a closed convex set in an arbitrary Riemannian manifold $(M,g)$ and $\gamma$ be a geodesic in the boundary of $\mathfrak{C}$ equipped with intrinsic metric.
Suppose that all points on $\gamma$ have rank at most 1 in $\mathfrak{C}$.
Then $\gamma$ is a geodesic of the ambient space $(M,g)$.
\end{thm}

One can show that generic Riemannian manifold contains convex closed sets with geodesics on its boundary.
For example almost horizonal geodesics in a perturbed product metric on $\SS^2\times\RR$ run in a convex surfaces.

\begin{thm}{Open question}
Is it true that the set of geodesics on convex hypersurfaces in generic Riemannian manifolds is locally finite.
\end{thm}

There is a chance that presented argument might lead to a counterexample to the following well known conjecture:

\begin{thm}{Conjecture}
Let $X$ be a complete length $\CAT(0)$ space (not necessary locally compact).
Then any compact set lies in a compact convex subset of $X$.
\end{thm}

\begin{thm}{Open question}
Let $\mathfrak{C}$ be the closure of a convex hull of a set $Q$ in a Riemannian manifold.
Suppose $\gamma$ is a maximal $g$-geodesic that runs in $\partial \mathfrak{C}$.
Is it true that the ends of $\gamma$ belong to $Q$?
\end{thm}


\bibliographystyle{ieeetr}
\bibliography{convex}
\end{document}
