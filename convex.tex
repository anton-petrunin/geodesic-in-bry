\documentclass[a4paper,10pt]{article}
\usepackage{paper}
\hypersetup{pdftitle={Geodesics that run in convex surfaces}%
,pdfauthor={Alexander Lytchak and Anton Petrunin}
}
\begin{document}
 
\title{Peculiar properties of convex sets in generic Riemannian manifolds}
\author{Alexander Lytchak and Anton Petrunin}
\date{}
\maketitle

\begin{abstract}
We give a necessary condition on a geodesic in a Riemannian manifold that can run in some convex hypersurface.
As a corollary we obtain a number of peculiar properties of \emph{all} convex sets that holds true in \emph{generic} Riemannian manifolds $M$ of dimension at least 3.
In particular, the boundary of any convex set in $M$ contains at most countable set of geodesics.
\end{abstract}

\section{Introduction}
Let $\mathfrak{C}$ be the convex hull of a compact subset $A$ in a Euclidean space.
By Carathéodory's theorem, any point in $\mathfrak{C}\setminus A$ is an inner point of a geodesic contained in $\mathfrak{C}$.
In particular, if $\mathfrak{C}$ has a non-empty interior, then $\partial \mathfrak{C} \setminus A$ is  a union of geodesics.  
These well-known statements admit straightforward generalizations to the sphere and the hyperbolic space.
 
The above statement is also easy to prove in all simply connected non-positively curved surfaces and locally in any two-dimensional Riemannian manifold.
Recall that a set $\mathfrak{C}$ in a Riemannian manifold $M$ is called \emph{convex} if for any pair of points $x,y\in \mathfrak{C}$ any minimizing geodesic $[x,y]$ lies in $\mathfrak{C}$.
A point $x\in \mathfrak{C}$ is called \emph{extremal} if it does not lie in an interior of a geodesic in $\mathfrak{C}$.

It seems to be a folklore belief that a version of this statement should hold true in all Riemannian manifolds. 
For instance, it is claimed by Mohammad Ghomi and Joel Spruck \cite[Lemma 9.1]{Ghomi}.
In the present note we prove that the somewhat counter-intuitive opposite is the case for \emph{generic} Riemannian manifolds. 
Following \cite{eliashberg-mishachev}, we say that a \emph{generic Riemannian metric on a given class on manifold $M$ 
has a property $P$} if the set of metric tensors $g$ that satisfy $P$ is open and dense in the space of all Riemannian metric tensors of the given class.
Since the space of Riemannian metrics is \emph{Baire}, we may say that most of Riemannian metrics are generic.


\begin{thm}{Theorem}
Let $M$ be a smooth manifold with dimesnion at least 3.
Then, generic smooth Riemannian metric $g$ on $M$ satisfies the following properties:

\begin{subthm}{}
The set of extremal points of any closed convex set $\mathfrak{C}\subset (M,g)$ is dense in~$\partial\mathfrak{C}$
\end{subthm}

\begin{subthm}{}
Any convex set $\mathfrak{C}\subset (M,g)$ either is a subset of a geodesic, or has nonempty interior.
\end{subthm}


\begin{subthm}{}
Let $A_1$ be any compact non-convex  subset of $(M,g)$, not contained in a minimizing  geodesic. Define  $A_k$ inductively as  the union of all minimizing geodesics connecting pairs of  points in 
$A_{k-1}$.
Then, $A_{k+1}\supsetneq A_k$ for all $k$;
that is, $A_k$ is not convex for any $k$.
\end{subthm}


\begin{subthm}{}
For any convex closed set $\mathfrak{C}\subset (M,g)$ at most countable set of geodesics of $(M,g)$ contain all the geodesics that run in~$\partial\mathfrak{C}$.
\end{subthm}
\end{thm}

The proofs of all these statements are built on the key lemma stated in the following section; 
it describes a necessary condition on a geodesic in a Riemannian manifold that can run in some convex hypersurface.
The remaining part is nearly self-evident;
its formal proofs use Thom transversality theorem, and it takes the remaining sections.

\section{Key lemma}


Its formulation uses the notion of \emph{tangent cone} to a convex set which we are about define.

Given a point $x$ in a closed convex set $\mathfrak{C}$, define its tangent cone $\K_x\mathfrak{C}$ as the closure of subset of tangent space $\T_x$ formed by all velocity vectors of geodesics that start at $x$ and run in $\mathfrak{C}$.
Note that if $x$ lies in the interior of $\mathfrak{C}$, then $\K_x\mathfrak{C}=\T_x$.

Given a tangent vector $X\in\T_pM$, set
\[R(X)\:V\mapsto \Rm(V,X)X,\]
where $\Rm$ denotes the curvature tensor of $M$.
Note that $R(X)$ is a self-adjoint operator $\T_p\to \T_p$ and the Jacobi equation along a geodesic $\gamma$ can be written as 
\[I''+R(\gamma')\cdot I=0,\]
here we write $I''$ as a shortcut for $\nabla^2_{\gamma'}I$.
 
\begin{thm}{Key lemma}\label{lem:key}
Let $(M,g)$ be a Riemannian manifold and $\gamma$ is a geodesic in $(M,g)$ that is defined on an open interval.
Suppose that $\gamma$ lies in a closed convex set $\mathfrak{C}\subset (M,g)$.
Then the tangent cone of $\mathfrak{C}$ is parallel along $\gamma$; that is, the parallel translation along $\gamma$ defines a bijection between the tangent cones $\K_{\gamma(a)}\mathfrak{C}$ and $\K_{\gamma(b)}\mathfrak{C}$.

Moreover, for any $V\in \K_x\mathfrak{C}$ we have
\[R(\gamma')\cdot V\in \K_V[\K_x\mathfrak{C}].\]
In particular, if $E(t)$ denotes the maximal Euclidean subspace of $\K_{\gamma(t)}\mathfrak{C}$, then $E(t)$ is an invariant subspace of $R(\gamma'(t))$ for any $t$.
\end{thm}

The next observation follows immediately from the fact that Jacobi fields describe the difference between infinitesimally close geodesics.

\begin{thm}{Observation}
Let  $\mathfrak{C}$ be a closed convex subset in a Riemannian manifold $(M,g)$
Suppose that $\gamma\:(0,1)\to\mathfrak{C}$ be a geodesic in $(M,g)$ and $I$ is a Jacobi field along $\gamma$.
Assume that $I(t_0)\in \K_{\gamma(t_0)}\mathfrak{C}$ and $I(t_1)\in \K_{\gamma(t_1)}\mathfrak{C}$ for some $0<t_0<t_1<1$.
Then $I(t)\in \K_{\gamma(t)}\mathfrak{C}$ for any $t\in [t_0,t_1]$.
\end{thm}

\parit{Proof of \ref{lem:key}.} 
Choose a subinterval $[a,b]$ in the domain of $\gamma$.
Given a large positive integer $k$, consider arithmetic progression
$(t_i)$ such that $t_0=a$, $t_k=b$.

Choose a tangent vector $I_0\in\T_{\gamma(a)}$. 
Consider the sequence of vectors $I_i\z\in\T_{\gamma(t_i)}$ defined the following way: $I_{i+1}=2\cdot I_i(t_{i+1})$ where $I_i(t)$ denotes the Jacobi field along $\gamma$ such that $I_i(t_i)=I_i$ and $I_i(t_{i+2})=0$.

Set $\iota_k(I_0)\df I_k$;
By the observation, if $I_0\in \K_{\gamma(a)}$ then $\iota_k(I_0)\in \K_{\gamma(b)}$.
Note that $\iota_k(I_0)$ converges to the parallel translation of $I_0$ along $\gamma$ as $k\to \infty$.
It follows that parallel translation along $\gamma$ maps $\K_{\gamma(a)}\mathfrak{C}$ in $\K_{\gamma(b)}\mathfrak{C}$.
Swapping $a$ and $b$, shows that the same holds the other way around.
That is, the tangent cone $K_\gamma(t)\mathfrak{C}$ is parallel along $\gamma$.

Let us use parallel translation along $\gamma$ to identify the tangent spaces at the points on $\gamma$.
This way we will identify the tangent cones $K_{\gamma(t)}\mathfrak{C}$ for all $t$;
denote the obtained cone by $K_\gamma\mathfrak{C}$.

Choose $I_0\in K_\gamma\mathfrak{C}$ and $\eps>0$;
consider the Jacobi field $I$ such that $I(t_0+\eps)\z=I(t_0+\eps)=I_0$.
By Jacobi equation, we have
\[I(t_0)=I_0+\eps^2\cdot R(\gamma'(t_0))\cdot I_0+o(\eps^2).\]
According to the observation, $I(t_0)\in \K_{\gamma} \mathfrak{C}$.
Since $\K_{\gamma} \mathfrak{C}$ is closed, $R(\gamma')\cdot I\z\in \K_I\K_\gamma\mathfrak{C}$.

Finally note that $I\in E(t_0)$ if and only if $\K_I\K_{\gamma(t_0)}\mathfrak{C}=\K_{\gamma(t_0)}\mathfrak{C}$.
Therefore, if $I\in E(t_0)$, then $\pm R(\gamma'(t_0))\cdot I\in \K_{\gamma(t_0)}\mathfrak{C}$ and hence $R(\gamma'(t_0))\cdot I\in E(t_0)$.
That is $E(t)$ is invariant subspace of $ R(\gamma'(t))$ for any $t$.
\qeds
  
\section{Jets of metric tenors}\label{sec:jet}

Let $(M,g)$ be a Riemannian manifold.
Recall that $\Rm$ denotes the curvature tensor.

Choose a tangent vector $X_p\in \T_pM$; 
extend $X_p\in \T_p$ to a parallel vector field $X$ in the normal coordinates at $p$.
That is, set 
\[X_{\exp_pW}=(d_W\exp_p)X_p\] for all small $W\in \T_p$.
Note that $(\nabla^k_X X)_p=0$ for any $k\ge 1$.

Recall that 
\[R=R(X)\:V\mapsto \Rm(V,X)X.\]
Given a tensor field $S$, let us use the shortcut $S'$  for $\nabla_XS$ at $p$.
In particular, set
\[R^{(k)}=R^{(k)}(X)\df\nabla^k_XR(X).\]
Note that 
\begin{itemize}
\item $R^{(k)}(X)\:\T_p\to\T_p$ is a self-adjoint operator and
\item $X\mapsto R^{(k)}(X)$ is a homogenius polynomial of degree $k+2$ defined for all $X\in \T_p$.
\end{itemize}

 


\begin{thm}{Lemma}\label{lem:jacobi}
For any tangent vector $X$ at a point $p$ of a Riemannian manifold manifold $M$ and any integer $n$ there are three self-adjoint operators 
\[A_n=A_n(X),\ B_n=B_n(X),\ C_n=C_n(X)\:\T_p\to \T_p\]
that satisfy the following conditions:

\begin{subthm}{lem:jacobi:ABC}
For any Jacobi field $I$ along the geodesic $\gamma\:t\mapsto \exp_p(t\cdot X)$ the following identity holds:
\[\langle I,I\rangle^{(n)}
=
\langle A_n\cdot  I,I\rangle+ \langle B_n\cdot I, I'\rangle+\langle C_n\cdot I', I'\rangle.\]
\end{subthm}

\begin{subthm}{lem:jacobi:R}
 $X\mapsto A_{n}(X)$, $X\mapsto B_{n+1}(X)$ and $X\mapsto C_{n+2}(X)$ are homogeneous polynomials of degree $n$ defined for all $X\in \T_p$.
\end{subthm}

\begin{subthm}{lem:jacobi:A-R}
The following differences  
\[A_{n+2}-2\cdot R^{(n)},
\quad 
B_{n+3}-4\cdot(n+2)\cdot R^{(n)},
\quad
C_{n+4}-2\cdot(n+1)\cdot(n+4)\cdot R^{(n)}\]
depend only on $R,R',\dots, R^{(n-2)}$.
\end{subthm}


\end{thm}

\parit{Proof.}
Applying recursively the Jacobi identity $I''=-R\cdot I$ for the geodesic $\gamma\:t\mapsto\exp_p(t\cdot X)$, we get that the following identity 
\[\langle I,I\rangle^{(n)}
=
\langle A_n\cdot  I,I\rangle+ \langle B_n\cdot I,I'\rangle+\langle C_n\cdot I',I'\rangle\]
where $A_0=1$, $B_0=0$, $C_0=0$, and the operators $A_n$, $B_n$, and $C_n$ satisfy the following recursive relations:
\begin{align*}
A_{n+1}&=A_n'-\tfrac12\cdot R\cdot B_n-\tfrac12\cdot B_n\cdot R
\\
B_{n+1}&=B_n'+ 2\cdot A_n-C_n\cdot R-R\cdot C_n
\\
C_{n+1}&=C_n'+ B_n
\end{align*}

\renewcommand{\arraystretch}{1.5}
\begin{figure}[!ht]
\centering
\begin{tabular}{ l|c|c|c }
$n$ & $A$ & $B$ & $C$ \\ \hline
0& $1$   &  $0$  & $0$ \\ \hline
1& $0$   &  $2$  & $0$ \\ \hline 
2& $-2\cdot R$ & $0$ & $2$ \\ \hline 
3& $-2\cdot R'$ & $-8\cdot R$ & 0  \\ \hline 
4& $-2\cdot R''+8\cdot R^2$ & $-12\cdot R'$ & $-8\cdot R$  \\ \hline
5& $-2\cdot R^{(3)}+\dots$ 
& $-16\cdot R''+16\cdot R^2$ & $-20\cdot R'$  \\ \hline
6
&$-2\cdot R^{(4)}+\dots$
&$-20\cdot R^{(3)}+\dots$
&$-36\cdot R''+16\cdot R^2$
\\
\end{tabular}
\end{figure} 

It proves part \ref{SHORT.lem:jacobi:ABC};
by looking at the table, and applying induction, one gets~\ref{SHORT.lem:jacobi:R} and~\ref{SHORT.lem:jacobi:A-R}.\qeds

Following \cite{eliashberg-mishachev}, we will denote by $J^n_g(p)$ the $n$-jet of a metric tensor $g$ at the point $p$;
it cares information on all partial derivatives of $g$ at $p$ up to order~$n$ in some (and therefore every) local coordinate system at $p$.
Equivalently, the jet $J^n_g(p)$ can be described by Taylor expansion up to degree $n$ of $g$ at $p$ (in some, and therefore every, local coordinate system at $p$).

Assume $g$ is positive definite at $p$.
Note that $J^n_g(p)$ determines the polynomials 
$X\mapsto R(X)$, $X\mapsto R'(X),\dots, X\mapsto R^{(n-2)}(X)$ defined for all $X\in \T_p$. 
In other words, the map 
\[\rho_n\:J^n_g(p)\mapsto (R,\dots,R^{(n-2)})\]
is defined.
The domain $\Dom \rho_n\subset J^n_g(p)$ of $\rho_n$ is the open set of all $n$-jets such that $g$ is positive definite at $p$.

\begin{thm}{Proposition}\label{prop:submersion}
The map $\rho_n$ is a submersion;
that is, for any jet $J^n_g(p)\z\in \Dom \rho_n$ the differential $d_{J^n_g(p)}\rho_n$ has a right inverse.
\end{thm}

\parit{Proof.}
Choose a Riemannian metric $g$ in a neighborhood of $p$.
Let $C_i=C_i(X)$ be as in \ref{lem:jacobi}.

Choose another vector filed $V$ that is parallel in the normal coordinates at~$p$.

Set $V_t=V(\exp_p(t\cdot X))$ and $I_t=t\cdot V_t$.
Note that $I$ is a Jacobi field along $\gamma$.
Observe that $t^2\cdot \langle V_t, V_t\rangle=\langle I_t, I_t\rangle$, $I(0)=0$, and $I'(0)=V(0)$.
Applying \ref{lem:jacobi}, we get
\[g(I_t, I_t)=\sum_{i=0}^{n+2}\tfrac1{(n+2)!}\cdot\langle C_n(X)\cdot V_0,V_0\rangle\cdot t^{n+2}+o(t^{n+2}).\]
Since $X$ is arbitrary, we get the equality
\[g( V, V)=\sum_{i=2}^{n+2}\tfrac1{(n+2)!}\cdot\langle C_n(X)\cdot V_0,V_0\rangle+o(|X|^{n})\]
at the point $\exp_p X$.
It follows that in the normal coordinates at $p$,
the $n$-jet $J^n_g(p)$ is uniquely described by the polynomials
\[X\mapsto C_2(X),
\ \dots\ ,\  
X\mapsto C_{n+2}(X).\]

Applying \ref{lem:jacobi:A-R} inductively on $n$, we get that the differential 
$d_{J^n_g(p)}\rho_n$ has a right inverse $s$.
Moreover $s$ can be chosen so that the jets in its image correspond to metric tensors that share the normal coordinates with $g$.
Hence the proposition follows.
\qeds 

\section{Genericity}

\begin{thm}{Proposition}
Let $M$ be a smooth manifold of dimension $n\ge 3$.
Then a generic Riemannian metric $g$ on $M$ satisfies the following property:
There is no tangent vector $X\ne 0$ such that the operators
$R(X), R'(X)$, and $R''(X)$ have a common eigenvector.
\end{thm}

The proposition follows from the Thom transversality theorem \cite[2.3.2]{eliashberg-mishachev}, \ref{prop:submersion}, and the following lemma.


\begin{thm}{Lemma}
Let $S^k$ be the space of all $k$-tuples $A_1,\dots A_k$ of self-adjoint operators $\RR^n\to\RR^n$.
Then the set $E$ of all $k$-tuples $(A_1,...A_k) \in S^k$  having a common nontrivial invariant subspace is an algebraic subset of $S^k$ of codimension $(m-1)\cdot (k-1)$.
\end{thm}

\parit{Proof.}\qeds


\bibliographystyle{alpha}
\bibliography{convex}
\end{document}
