\documentclass[a4paper,10pt]{article}
\usepackage{paper}

\def\thetitle{About every convex set in\\ generic Riemannian manifold}
\hypersetup{
pdftitle={\thetitle},
pdfauthor={Alexander Lytchak and Anton Petrunin}
}

\begin{document}
%\pagestyle{empty}

\title{\thetitle}
\author{Alexander Lytchak and Anton Petrunin}
\date{}
\maketitle

\begin{abstract}
We give a necessary condition on a geodesic in a Riemannian manifold that can run in some convex hypersurface.
As a corollary we obtain peculiar properties that hold true for \emph{every} convex set in any \emph{generic} Riemannian manifold $(M,g)$.
For example, if a convex set in $(M,g)$ is bounded by a smooth hypersurface, then it is strictly convex.
\end{abstract}

%keywords: Convex subset, generic Riemannian manifold, convex hull, extreme points
%MathClassification:  53C22   (secondary:    53C29,  53C21, 52A99).


\section*{Preface}



Consider a regular triangle in the $\RR^3$ with vertices $x$, $y$ and $z$.
Let $g$ be a Riemannian metric on $\RR^3$ that is obtained by a tiny smooth generic perturbation of the canonical metric.
Denote by $\Delta$ the \emph{convex hull} of $\{x,y,z\}$ in $(\RR^3,g)$; that is, the smallest set that contains the vertices and a minimizing geodesic between any pair of its points.

\smallskip

\emph{Is it always true that a geodesic $[x y]$ lies on the boundary of $\Delta$?}

\smallskip

Try to think about this question for some time, and then read further.

\section{Introduction}
Let $\mathfrak{C}$ be the convex hull of a subset $Q$ in a Euclidean space.
By Carathéodory's theorem, any point in $\mathfrak{C}\setminus Q$ is an inner point of a line segment contained in $\mathfrak{C}$;
that is, the complement $\mathfrak{C} \setminus Q$ does not contain extreme points of $\mathfrak{C}$.
This statement admits a straightforward generalization to the sphere and the  Lobachevsky space; moreover, it  is also easily seen to hold \emph{locally} in any two-dimensional Riemannian manifold.

Recall that a set $\mathfrak{C}$ in a Riemannian manifold $(M,g)$ is called \emph{convex} if for any pair of points $x,y\in \mathfrak{C}$ any minimizing geodesic $[x,y]$ lies in $\mathfrak{C}$.
A point in $\mathfrak{C}$ is called \emph{extreme} if it does not lie in an interior of a geodesic in~$\mathfrak{C}$.

It seems to be a folklore belief that a version of the statement above should hold true in all Riemannian manifolds.
For instance, it is claimed by Mohammad Ghomi and Joel Spruck \cite[Lemma 9.1]{Ghomi}.
In the present note we prove that the somewhat counter-intuitive opposite is the case for \emph{generic} Riemannian manifolds.
It agrees with the pattern: \emph{a typical object in your favorite theory looks like nothing you have ever seen before}.

Further Riemannian manifolds will be assumed to be connected and $\mathcal C^\infty$-smooth.
Given a positive integer $k$, we say that a property $\mathcal P$ holds for \emph{$\mathcal C^k$-generic} Riemannian metric $g$ on a manifold $M$ 
if the property $\mathcal P$ holds for a dense \emph{G-delta set} (that is, a countable intersection of open subsets) of metric tensors in the $\mathcal C^k$-topology.

\begin{thm}{Main theorem}\label{thm:main}
Let $\mathfrak C$ be an arbitrary convex subset of a $\mathcal C^2$-generic Riemannian manifold $(M,g)$.
Then the set of non-extreme points in $\mathfrak C$ is the union of an open set and at most countable family of geodesics in $(M,g)$. 

In particular, if $\dim M\ge 3$, and $\mathfrak C$ is not contained in a geodesic then the set of
extreme points of $\mathfrak{C}$ is dense in~$\partial\mathfrak{C}$.
\end{thm}

If $\dim M \le 2$, then the statement is rather trivial and holds true  for all Riemannian metrics, not only for generic ones.
For $\dim M \ge  3$
the theorem implies the following corollary.
In case $\dim M=3$ its proof has been sketched by Robert Bryant \cite{Bryant}; for $\dim M\ge 4$ it was proved by Thomas Murphy and Frederick Wilhelm \cite{Wilhelm}.

\begin{thm}{Corollary}\label{cor:main}
Let $(M,g)$ be a $\mathcal C^2$-generic Riemannian manifold.
Then any convex subset $\mathfrak C$ of $(M,g)$ is either contained in a geodesic
or \emph{full-dimensional}; that is, the interior of $\mathfrak C$ is nonempty.
\end{thm}

Recall that the \emph{convex hull} of a set $Q$ is defined as the minimal convex set (with respect to inclusion) that contains $Q$.
According to Carathéodory's theorem, the convex hull of  any compact set of a Euclidean space (as well as sphere or the Lobachevsky space) is compact.
According to the following corollary, the latter never holds in most Riemannian manifolds.

\begin{thm}{Corollary}\label{cor:caratheodory}
Let $Q$ be a closed subset of a $\mathcal C^2$-generic Riemannian manifold $(M,g)$ of dimension at least $3$. If $Q$ does not lie in a  geodesic and
the convex hull $\mathfrak C$ of $Q$ is closed then $\partial \mathfrak C \subset Q$.
\end{thm}

In the case that the  set $Q$ consist of $3$ points, this Corollary provides a negative answer to a question formulated by Marcel Berger \cite[Note 6.1.3.1]{berger-2003}.

The proofs are built on the following proposition.
Its formulation uses the notion of \emph{rank} of a point $p$ in a closed convex set $\mathfrak{C}$;
we define it as the dimension of maximal linear subspace in the tangent cone to $\mathfrak{C}$ at $p$.

\begin{thm}{Main proposition}\label{prom:rank}
Suppose $\mathfrak{C}$ is a closed convex set in a $\mathcal C^2$-generic $m$-dimensional Riemannian manifold $(M,g)$.
Then all non-extreme points of $\mathfrak{C}$ have rank either $1$ or $m$.

In particular, if $\dim M\ge 3$ and $\mathfrak{C}$ is bounded by a $\mathcal{C}^1$-smooth hypersurface, then $\mathfrak{C}$ is \emph{strictly convex};
that is, all boundary points of $\mathfrak{C}$ are extreme.
\end{thm}

A  weaker version of this proposition was sketched by the second author~\cite{petrunin-2009}.  

The proof of the proposition is built on the key lemma stated in the following section;   
it describes a necessary condition on a geodesic in a Riemannian manifold that stays
% can run 
in some convex, possibly non-smooth, hypersurface.
If the geodesic contains a point of rank at least 2, then this condition implies a non-trivial
statement on the curvature tensor of the manifold.
Then we show that the curvature tensor of a generic Riemannian manifold does not meet this property.
The latter part is technical but straightforward; it is done by applying the Thom transversality theorem; see Appendix \ref{sec:normalization}.

In Appendix \ref{app:remarks} we list a few related open questions.

\parbf{Acknowledgments.}
We thank Mohammad Ghomi for his interest and communications.
Alexander Lytchak was partially supported by the DFG grant, no. 281071066, TRR 191.
Anton Petrunin was partially supported by the NSF grant, DMS-2005279.



\section{Key lemma}\label{sec:key}

Let $\mathfrak{C}$ be a closed convex set in an $m$-dimensional Riemannian manifold $(M,g)$.
Recall that $\T_x=\T_xM$ denotes the \emph{tangent space} of $M$ at $x$.
The \emph{tangent cone} $\K_x=\K_x\mathfrak{C}\subset \T_x$ at $x\in\mathfrak{C}$ is defined as the closure of the set of all velocity vectors of geodesics that start at $x$ and run in $\mathfrak{C}$.

Given $x\in \mathfrak{C}$, denote by $\L_x=\L_x\mathfrak{C}$ the \emph{maximal linear subspace} of $\K_x$.
We define the \emph{rank} of $x$ in $\mathfrak{C}$ as the dimension of $\L_x$.

Note that $\K_x$ is a convex cone in $\T_x$; in particular, $\L_x=\K_x\cap (-\K_x)$.
Further $\K_x$ coincides with 
$\T_x$ if and only if
a neighborhood of $x$ lies in the interior of $\mathfrak{C}$.
In other words, $x$ has rank $m$ if and only if $\mathfrak{C}$ contains
a neighborhood of $x$.


Given a tangent vector $\vec x\in\T_pM$, consider the  \emph{Jacobi operators} of order $k$
\[R^k_\vec x\:\vec{v}\mapsto \nabla^{k-2}_\vec x\Rm(\vec{v},\vec x)\vec x,\]
where $\Rm$ denotes the curvature tensor of $g$;
we set $R^1=0$.
Note that (i) $R^k_\vec x\:\T_p\z\to \T_p$ is a self-adjoint operator, (ii) $\vec x\mapsto R^k_\vec x$ is a homogeneous polynomial of degree $k$, and (iii) 
\[R^k_\vec x\cdot\vec x=0 \eqlbl{eq:RXX=0}\]
for any $k$ and $\vec x\in\T_p$.

The Jacobi equation along a geodesic $\gamma$ takes the form 
\[\nabla^2_{\gamma'}\cdot\vec i+R^2_{\gamma'}\cdot \vec i=0.\]

\begin{thm}{Key lemma}\label{lem:key}
Let $(M,g)$ be a Riemannian manifold and $\gamma\:(a_0,b_0)\to M$ be a geodesic that runs in a closed convex set $\mathfrak{C}\subset (M,g)$.
Then the tangent cones of $\mathfrak{C}$ are parallel along $\gamma$; that is, the parallel translation along $\gamma$ defines a bijection between the tangent cones $\K_{\gamma(a)}\mathfrak{C}$ and $\K_{\gamma(b)}\mathfrak{C}$ for any $a,b \in (a_0,b_0)$.

Moreover, for any $a\in (a_0,b_0)$ the following conditions  hold:
\begin{subthm}{lem:key:a}
For any $\vec{v}\in \K_{\gamma(a)}\mathfrak{C}$ we have
\[R^2_{\gamma'(a)}\cdot \vec{v}\in \K_\vec{v}[\K_{\gamma(a)}\mathfrak{C}].\]
\end{subthm}

\begin{subthm}{lem:key:b} 
$\L_{\gamma(a)}\mathfrak{C}$ is an invariant subspace of $R^2_{\gamma'(a)}\:\T_{\gamma(a)}\to\T_{\gamma(a)}$.
\end{subthm}

\end{thm}

The proof uses the fact that the parallel translation can be defined via geodesics.
In a similar way, this observation was used in \cite[Section 13]{Ber-Nik} and~\cite{Petruninpar}.
In fact the main part of the key lemma follows from
\cite{Petruninpar}.

\parit{Proof of \ref{lem:key}.} 
Since all statements are local, we may replace $(M,g)$ by its small open convex subset.
By doing so we may assume that any pair of points of $(M,g)$ is connected by a unique geodesic and there are no conjugate points.
In particular, for any subinterval $[a,b]\subset (a_0,b_0)$ and any tangent vectors $\vec{v} \in \T_{\gamma (a)}$ and $\vec{w} \in \T_{\gamma (b)}$ there exists unique Jacobi field $\vec i$ along $\gamma$ 
such that $\vec i(a)=\vec{v}$ and~$\vec i(b)=\vec{w}$.

Since Jacobi fields are variational fields of geodesic variations, 
the convexity of $\mathfrak{C}$ implies the following: 

\begin{thm}{Observation}
Suppose $\vec i$ is a Jacobi field along % a geodesic
$\gamma$ and $a_0<a<t<b<b_0$.
If 
$\vec i(a)\in \K_{\gamma(a)}\mathfrak{C}$ and $\vec i(b)\in \K_{\gamma(b)}\mathfrak{C}$
then $\vec i(t)\in \K_{\gamma(t)}\mathfrak{C}$.
\end{thm}

Choose a subinterval $[a,b] \subset (a_0,b_0)$.
Given a large positive integer $k$, consider the arithmetic progression
$t_0,\dots,t_{k+1}$ such that $t_0=a$ and $t_k=b$.

Choose a tangent vector $\vec{v}_0\in\T_{\gamma(a)}$.
Consider the sequence of vectors $\vec{v}_i\z\in\T_{\gamma(t_i)}$ defined recursively by $\vec{v}_{i+1}=2\cdot \vec i_i(t_{i+1})$, where $t\mapsto \vec i_i(t)$ denotes the Jacobi field along $\gamma$ such that $\vec i_i(t_i)=\vec{v}_i$ and $\vec i_i(t_{i+2})=0$.

\begin{figure}[ht!]\vskip-0mm\centering\includegraphics{mppics/pic-1}\end{figure}

Define $\iota_k\:\T_{\gamma(a)}\to \T_{\gamma(b)}$ by setting $\iota_k(\vec v_0)\df \vec v_k$.
According to the observation, if $\vec v_0\in \K_{\gamma(a)}\mathfrak{C}$, then $\iota_k(\vec v_0)\in \K_{\gamma(b)}\mathfrak{C}$.
As observed in \cite{Ber-Nik} and~\cite{Petruninpar}, $\iota_k(\vec v_0)$ converges to the parallel translation of $\vec v_0$ along $\gamma$ as $k\to \infty$.
Since $\K_{\gamma(b)}\mathfrak{C}$ is closed,
the parallel translation along $\gamma$ maps $\K_{\gamma(a)}\mathfrak{C}$ in $\K_{\gamma(b)}\mathfrak{C}$.
Switching the direction of $\gamma$, we get the opposite inclusion.
That is, the tangent cones $\K_{\gamma(t)}\mathfrak{C}$ are parallel along $\gamma$ --- the main part is proved.

Let us use the parallel translation along $\gamma$ to identify the tangent spaces at points on $\gamma$.
This way we identify the tangent cones $\K_{\gamma(t)}\mathfrak{C}$ for all $t$;
denote the obtained cone by $\K$.

For $\vec{v}\in \K$ and small $\epsilon>0$, consider the unique Jacobi field $\vec i_\epsilon$ along $\gamma$ with $\vec i_\epsilon (a+\eps)\z=\vec i _\epsilon(a-\eps)=\vec{v}$.
Due to the Jacobi equation,
\[\vec i_\epsilon (a)=\vec{v} +\eps^2\cdot R^2_{\gamma'(a)}\cdot \vec{v} +o(\eps^2).\]
According to the observation, $\vec i_\epsilon(a)\in \K$ for any $\eps>0$.
Since $\K$ is a closed convex cone, we get $R^2_{\gamma'}\cdot \vec{v}\z\in \K_\vec{v}\K$ --- \ref{SHORT.lem:key:a} is proved.

Finally,
$\vec{v}\in \L_{\gamma(a)}\mathfrak{C}$ $
\iff$
$\vec{v}, -\vec{v}\in \K$
$\iff$
$\K_\vec{v}\K=\K$.
Therefore, if $\vec{v}\in \L_{\gamma(a)}\mathfrak{C}$, then $\pm R^2_{\gamma'(a)}\cdot \vec{v}\in \K$, and hence $R^2_{\gamma'(a)}\cdot \vec{v}\in \L_{\gamma(a)}\mathfrak{C}$.
That is, $\L_{\gamma(a)}\mathfrak{C}$ is an invariant subspace of $R^2_{\gamma'(a)}$ --- \ref{SHORT.lem:key:b} is proved.
\qeds

\section{Main proposition}

In this section we will prove the main proposition \ref{prom:rank} modulo one claim; let us introduce notations to state the needed claim.

Let $M$ be a smooth $m$-dimensional manifold with a Riemannian metric $g$.
Recall that $R^k_\vec x\:\T_p\to\T_p$ denotes the Jacobi operators of $g$ of order $k$ for a tangent vector $\vec x\in\T_p$.

Suppose $\vec x$ is a nonzero tangent vector at a point $p\in M$.
An invariant subspace $V\subset \T_p$ of $R^k_\vec x$ will be called \emph{exceptional} if $V\ni \vec x$ and $1< \dim V<m$.
(Recall that $R^k_\vec x\cdot \vec x=0$
for any $k$ and $\vec x\in \T_p$.
Therefore the subspace spanned by $\vec x$ is always an invariant subspace of $R^k_\vec x$ for any $k$.)

We will  say that a metric $g$ on a manifold $M$ is $k$-exceptional if there exists a point $p\in M$ and a non-zero vector $\vec x\in T_p M$,
such that the operators   $R^2_\vec x,\dots, R^k _\vec x$ have a common exceptional invariant subspace.  

The following non-surprising but slightly technical statement will be derived in Appendix~\ref{sec:normalization} from Thom transversality theorem:

\begin{thm}{Claim}\label{clm:codim-sigma} 
For any smooth manifold $M$ there exists an integer $k$ such that the 
$\mathcal C^k$-generic Riemannian metric is not $k$-exceptional.
\end{thm}

For $k=2$  (and, probably,  also for $k=3$) every Riemannian metric is $k$-exceptional.  However, for large $k$, the
$k$-exceptionality clearly defines more and more  non-trivial restrictions on the curvature tensor. Therefore, it is not surprising that \emph{most}
Riemannian metrics are not $k$-exceptional, for sufficiently large $k$.
A formal, slightly technical proof of this claim will be derived in Appendix~\ref{sec:normalization}.

\parit{Proof of \ref{prom:rank} modulo \ref{clm:codim-sigma}.}
Suppose that $p$ is a nonextreme point of $\mathfrak{C}$;
that is, $p$ lies on a nonconstant geodesic $\gamma\:(a,b)\to\mathfrak{C}$.

According to the key lemma (\ref{lem:key}), the family of \emph{maximal linear subspaces} $\L_{\gamma(t)}\mathfrak{C}$ of $\K_{\gamma(t)}\mathfrak{C}$ is parallel along $\gamma$ and invariant for $R^2_{\gamma(t)}$.
Note that $\L_p$ is exceptional if and only if the rank of $p$ is neither $1$, nor $m$.

Further, if a nontrivial geodesic $\gamma$ admits a parallel family $\L_t\subset \T_{\gamma(t)}$ of exceptional invariant subspaces for all $R^2_{\gamma(t)}$, then we say that $\gamma$ is \emph{exceptional}.
So, it is sufficient to show that $\mathcal C^2$-generic Riemannian manifolds $(M,g)$ do not have exceptional geodesics.

Choose a compact subset $K\subset M$ and $\eps>0$.
Consider the set $Z(K,\eps)$ of all Riemannian metrics $g$ on $M$ such that there exists an exceptional geodesic $\gamma$ in $(M,g)$ that starts at a point in $K$ and has length $\eps$.
Observe that the geodesics and the curvature tensor depend continuously on the Riemannian metric in $\mathcal C^2$-topology.
Therefore the set $Z(K,\eps)$ is closed with respect to the $\mathcal C^2$-topology.

Suppose $\gamma$ is an exceptional geodesic that passes thru $p$ in the direction $\vec x$.
By taking covariant derivatives along $\gamma$, we get that the Jacobi operators $R^k_\vec x$ have a common exceptional invariant subspace $\L_p$, for all $k \geq 2$.
In other words, 
\[Z^k(K)\supset Z(K,\eps),\eqlbl{eq:Zk<Z(K,eps)}\]
where $Z^k(K)$ denotes the set of all smooth Riemannian metrics on $M$ such that for some $p\in K$ and $\vec x\in\T_p\setminus \{0 \}$ the operators $R^2_\vec x,\dots, R^k _\vec x$ have a common exceptional invariant subspace.

By the very definition of $Z^k(K)$, it is closed with respect to $\mathcal C^{k}$-topology on the space of all Riemannian metrics on $M$.
By Claim~\ref{clm:codim-sigma}, we can choose $k$ so that  $Z^k(K)$ is $\mathcal{C}^k$-meager for any $K$;
that is, its complement is a dense G-delta set in the space of all Riemannian metrics on $M$ with $\mathcal{C}^k$-topology.

Since $Z(K,\eps)$ is closed with respect to the $\mathcal C^2$-topology, \ref{eq:Zk<Z(K,eps)} implies that $Z(K,\eps)$ is $\mathcal{C}^2$-meager in the space of all Riemannian metrics on $M$.


Choose a nested sequence of compact sets $K_1\z\subset K_2\subset \dots$ that cover $M$ and set $\eps_n=\tfrac1n$.
Set 
\[Z(M)=\bigcup_n Z(K_n,\eps_n);\]
since $Z(K_n,\eps_n)$ is $\mathcal{C}^2$-meager for every $n$, so is $Z(M)$.

It remains to note that $g\in Z(M)$ if and only if $(M,g)$ has an exceptional geodesic.
\qeds


\section{Main theorem}

The following proposition is a special case of a result of Nan Li and Aaron Naber \cite[Theorem 1.6]{li-naber}.
It also can be deduced from the result of Luděk 
Zajíček~\cite{zajicek}.

\begin{thm}{Proposition}\label{prop:rectifiable}
Let $\mathfrak{C}$ be a closed convex set in a Riemannian manifold $(M,g)$.
Then the set of points in $\mathfrak{C}$ with rank at most $k$ is countably \emph{$k$-rectifiable};
that is, this set can be  covered by images of a countable set of Lipschitz maps $\RR^k\to (M,g)$.
In particular, this set contains at most countably many disjoint sets with positive $k$-dimensional Hausdorff measure.
\end{thm}

\parit{Proof of \ref{thm:main} and \ref{cor:main}.}
Assume $\mathfrak{C}$ is closed.
According to \cite[Theorem 1.6]{cheeger-gromoll}, a connected closed convex set $\mathfrak{C}$ in a Riemannian manifold $(M,g)$ is homeomorphic to a manifold with boundary, say~$\mathfrak{B}$.
Moreover, the complement $\mathfrak{C}\backslash \mathfrak{B}$ is a totally geodesic submanifold of $(M,g)$; denote its dimension by $d$.

The tangent cone $\K_p \mathfrak C$ at any $p \in \mathfrak C\setminus \mathfrak B$ is $d$-dimensional linear space.
By the main proposition (\ref{prom:rank}), $d=0, 1$, or $m$.
If $d=0$, then $\mathfrak{C}$ is a single point.
If $d=1$, then $\mathfrak C\setminus \mathfrak B$ is a  geodesic in $(M,g)$;
hence $\mathfrak C$ is contained in a  geodesic as well.
If $d=m$, then $\mathfrak C\setminus \mathfrak B$ is open in $M$; that is, $\mathfrak C$ is full-dimensional --- Corollary 
\ref{cor:main} is proved.

By the main proposition (\ref{prom:rank}) any non-extreme point $x\in \partial \mathfrak C$ has rank 1.
Thus, there is a unique line in $\K_x\mathfrak C$ and it is the tangent line of a geodesic $\gamma\subset\mathfrak C$ hat has $p$ as an inner point. 

Let us extend $\gamma$ to a maximal open interval, so that $\gamma$ stays in $\mathfrak C$;
note that $p$ uniquely defines $\gamma$. 
By the main statement of the key lemma, all points on $\gamma$ lie on $\partial \mathfrak C$.
By definition, all such geodesics consist of non-extreme points.

It gives a subdivision of non-extreme points of $\partial\mathfrak C$ into geodesics with positive lengths.
By \ref{prop:rectifiable}, there are only countably many such geodesics.

If $\mathfrak C$ is not closed, consider its closure $\bar {\mathfrak C}$.
Note that $\bar{\mathfrak C}$ is locally convex and the above arguments apply to closed locally convex subsets without changes.
Denote its boundary by $\bar{\mathfrak{B}}$.
Observe that any nonextreme point of $\mathfrak{C}$ is a  nonextreme point of $\bar{\mathfrak{C}}$ and $\bar{\mathfrak{C}}\backslash\bar{\mathfrak{B}}\subset\mathfrak{C}$ \cite[Lemma 1.5]{cheeger-gromoll}.
Hence the statement follows.
\qeds

\parit{Proof of \ref{cor:caratheodory}.}
Start with $Q=Q_0$ and define inductively $Q_{i+1}$ to be the union of all minimizing geodesics between pairs of points of $Q_i$.
Note that the union $\mathfrak{C}= \bigcup_{i} 
Q_i$ is the convex hull of $Q$.

Without loss of generality we can assume that $\mathfrak{C}$ is a proper subset of $M$; in particular $\partial\mathfrak{C}\ne \varnothing$.
Since $Q$ is not contained in a geodesic, by the main theorem, $\mathfrak{C}$ has a non-trivial interior.
By the construction of $\mathfrak{C}$ above, any point $x\in \mathfrak{C} \setminus Q$ is not an extreme point of~$\mathfrak{C}$.

Assume $\partial \mathfrak{C} \not\subset Q$.
By the main theorem, the topological manifold $\partial \mathfrak{C}$ is the union of the closed subset $Q\cap \partial \mathfrak{C}$ and a countable union of geodesics.
But $\partial \mathfrak{C} \setminus Q$ is an $(m-1)$-dimensional topological manifold.
In particular it is not a union of countably many disjoint simple curves --- a contradiction.
\qeds

\appendix

\section{Normalization of metrics}
\label{sec:normalization}

This appendix is devoted to the algebra of curvature tensor and its covariant derivatives that leads to a proof of Claim \ref{clm:codim-sigma}.

Choose an $m$-dimensional Euclidean space $\T$.
Denote by $\mathcal{S}$ the space of self-adjoint operators on $\T$.

Consider the space $\mathcal{G}$ of germs of Riemannian metrics on $\T$ at $0$ that coincide with the canonical metric at $0$.
Any germ in $\mathcal{G}$ can described by $\langle G\cdot \vec v,\vec w\rangle$ where $\vec x\mapsto G_\vec x$ is a smooth function $\T\to\mathcal{S}$ such that $G_0=\id$. 

The $k$-jet of $G$ is defined by the Taylor polynomial of $G$ of degree $k$
$$G_\vec x=\id + G^1_\vec x+\dots+G^k_\vec x + o( |\vec x|^k ),
\eqlbl{eq:k-jet}$$
where $\vec x\mapsto G^i_\vec x$ is a homogeneous polynomial $\T\to\mathcal{S}$ of degree $i$.

Note that every array of polynomials $G^1,\dots, G^k\:\T\to \mathcal{S}$ such that $\deg G^i\z=i$ appears in \ref{eq:k-jet} for the germ in $\mathcal{G}$ defined by 
\[G_\vec x=\id + G^1_\vec x+\dots+G^k_\vec x.
\eqlbl{eq:k-jet+}\]
The space of $k$-jets of germs in $\mathcal{G}$ will be denoted by~$\mathcal{G}^k$.

A germ in $\mathcal{G}$ will be called \emph{normal} if the standard coordinates on $\T$ coincide with normal coordinates of the germ in a neighborhood of the origin.
By the Gauss lemma, a germ defined by $G$ is normal if and only if 
\[G_\vec x\cdot \vec x=\vec x\eqlbl{eq:Gauss}\]
for all small $\vec x\in \T$.
The subspace of normal germs in $\mathcal{G}$ and their $k$-jets will be denoted by $\mathcal{N}$ and $\mathcal{N}^k$, respectively.

Suppose that $G$ describes a germ in $\mathcal{N}$
and $G^1,\dots, G^k$ be as in \ref{eq:k-jet}.
By \ref{eq:Gauss} 
\[G^i_\vec x\cdot \vec x=0
\eqlbl{eq:Gaussi}\] 
for any $i$.
Moreover, for an array of polynomials $G^1,\dots,G^k\:\T\to\mathcal{S}$ such that $G^i$ is homogeneous of degree $i$ and \ref{eq:Gaussi} holds for each $i$, the sum \ref{eq:k-jet+} defines a normal $k$-jet;
that is, \ref{eq:Gaussi} is the only condition on the normality of jets.

From \ref{eq:Gaussi} we get that $G^1=0$.
Equivalently, Christoffel symbols vanish in the normal coordinates.

Choose $\vec x\in \T$; denote by $\mathcal{S}_\vec x$ the subspace of $\mathcal{S}$ of the operators $S$ such that $S\cdot \vec x=0$.
By \ref{eq:Gaussi}, $G^i_\vec x\in \mathcal{S}_\vec x$ for any germ in $\mathcal{N}$.
The following claim says that $G^i_\vec x$ can be chosen arbitrary in $\mathcal{S}_\vec x$ for $i\ge 2$ and $\vec x\ne 0$.

\begin{thm}{Claim}\label{clm:allSX}
Given $\vec x\ne 0$ in $\T$ and a sequence of operators $A_2,\dots A_k\in \mathcal{S}_\vec x$ there is a germ $(G^1,\dots,G^k)$ in $\mathcal{N}^k$ such that $G^i_\vec x=A_i$ for any $i\ge 2$.
\end{thm}

\parit{Proof.}
Choose a nonzero vector $\vec y$ perpendicular to $\vec x$;
denote by $A$ the projection of $\T$ to the line spanned by $\vec y$.

Given $\lambda\in \RR$ and $i\ge 2$ there is a normal germ such that $G^i_\vec x=\lambda\cdot A$ and $G^j_\vec x=0$ for all $j\ne i$;
such examples can be found among the germs of products of a surface of revolution and Euclidean space.

It remains to add constructed germs, for appropriate choices of $\vec y$ and $\lambda$ and use that Equation \ref{eq:Gaussi} is linear.
\qeds


Suppose that a germ in $\mathcal{G}$ is described by $G\:\T\to\mathcal{S}$.
Consider its array of Jacobi operators $(R^1,\dots,R^k)$ at the origin;
recall that $R^1=0$.
The identities in Section~\ref{sec:key} imply that any such array $(R^1,\dots,R^k)$ belongs to the space $\mathcal{R}^k$ defined by the following conditions: (i)
each $R^i\:\T\to \mathcal{S}$ is a homogeneous polynomial,
(ii) $\deg R^i=i$,
and (iii) $R^i_\vec x\cdot \vec x=0$ for any $i$ and $\vec x\in\T$.
Note that these conditions are exactly the same as for $G^i$ in $\mathcal{N}^k$.
Therefore $\mathcal{R}^k$ can be identified with $\mathcal{N}^k$, but we will keep separate notations for them.
In particular we may assume that $R^1=0$.

The expression of the curvature tensor in terms of the metric and its derivatives defines a natural algebraic map 
$$\rho_k\:\mathcal{G}^k\to \mathcal{R}^k.$$

Suppose that $(G^1,\dots,G^k)\in \mathcal{N}^k$.
Then \[G^k=a_k\cdot  R^k + A^k,\eqlbl{eq:a+A}\]
where $a_k$ is a nonzero constant and $A^k$ is a field of self-adjoint operators that can be written as a polynomial of $R^1,\dots R^{k-2}$.
This statement follows easily from the formula derived by Old\v{r}ich Kowalski and Martin Belger \cite[Proposition 2.2]{kowalski-belger}.
(In fact $a_k=-2\cdot\tfrac{k-1}{k+1}$ and $A_k$ is a linear combination of compositions of $R^1,\dots R^{k-2}$, but we will not need it.)

Note that formula \ref{eq:a+A} can be reverted 
\[R^k=\tfrac 1{a_k}\cdot  G^k + B^k,\eqlbl{eq:a+B}\]
where $B^k$ is a field of self-adjoint operators that can be written as a polynomial of $G^1,\dots G^{k-2}$.

Therefore we get the following:


\begin{thm}{Claim}\label{clm:diff}
The restriction $\rho_k|_{\mathcal{N}^k}$ is an algebraic diffeomorphism $\mathcal{N}^k\leftrightarrow\mathcal{R}^k$.
\end{thm}

Applying \ref{clm:allSX}, we get the following:

\begin{thm}{Corollary}\label{cor:Rall}
Given $\vec x\ne 0$ in $\T$ and a sequence of operators $A_2,\dots A_k\in \mathcal{S}_\vec x$ there is a germ $\mathcal{N}^k$ with Jacobi operators $R^i_\vec x=A_i$ for any $i\ge 2$.
\end{thm}

\begin{thm}{Proposition}\label{prop:submersion}
The map $\rho_k:\mathcal{G}^k\to \mathcal{R}^k$ is an algebraic submersion.
\end{thm}

\begin{wrapfigure}{r}{34mm}
\vskip-4mm
\centering
\begin{tikzpicture}[scale=2]
\node (1) at (0,1) {$\mathcal{G}$};
\node (2) at (1.4,1){$\mathcal{N}$};
\node (11) at (0,0){$\mathcal{G}^k$};
\node (12) at (1.4,0) {$\mathcal{N}^k$};
\node (21) at (.7,-5/6) {$\mathcal{R}^k$};
\draw[>=latex, auto=right, loop above/.style={out=75,in=105,loop}, every loop,]
   (1) edge[bend left] node [swap] {$\nu$}  (2)
   (2) edge[bend left, left hook-stealth] node {$\iota$} (1)
   (1) edge (11)
   (2) edge (12)
   (11) edge[bend right] node  {$\rho_k$} (21)
   (11) edge [bend left] node [swap] {$\nu_k$} (12)
   (12) edge[bend left, left hook-stealth] node {$\iota_k$} (11)
    (21) edge[<->, bend right] node  {$\rho_k$} (12)
   ;
\end{tikzpicture}
\label{diagram-page}
\end{wrapfigure}

\parit{Proof.}
Evidently $\rho_k$ is algebraic.

Any germ in $\mathcal{G}$ becomes normal if the space $\T$ is reparametrized by its exponential map.
It defines the normalization map 
$\mathcal{G}\xrightarrow{\nu} \mathcal{N}$.

The maps $\mathcal{G}^k\xrightarrow{\rho_k}\mathcal{R}^k\z{\xleftrightarrow{\rho_k}}\mathcal{N}^k$ together with the forgetful maps $\mathcal{G}\to\mathcal{G}^k$ and $\mathcal{N}\to\mathcal{N}^k$  commute.
In particular we get a map $\mathcal{G}^k\xrightarrow{\nu_k}\mathcal{N}^k$ that commutes with the forgetful maps and the normalization $\nu$.

Note that the inclusion $\mathcal{N}\stackrel{\iota}{\hookrightarrow} \mathcal{G}$ is a right inverse of $\nu$.
Moreover by changing the parametrization on $\T$ to normal coordinates of a given germ $G$ in $\mathcal{G}$, we may assume that $G$ lies in the image of $\iota$.
Therefore there is a map $\mathcal{N}^k\stackrel{\iota_k}{\hookrightarrow} \mathcal{G}^k$ that is right inverse of $\nu$ such that its image contains any given jet $\mathcal{G}^k$.
Whence $\mathcal{G}^k\xrightarrow{\nu^k} \mathcal{N}^k$ is a submersion.

By \ref{clm:diff}, $\mathcal{N}^k\xleftrightarrow{\rho_k}\mathcal{R}^k$ is a diffeomorphism, hence the result.
\qeds

\parit{Proof of \ref{clm:codim-sigma}.}
Denote by $\tilde{\mathcal{G}}^k$ the space of $k$-jets of Riemannian metrics at a given point $p$.
Denote by $\tilde\Sigma^k$ all jets in $\tilde{\mathcal{G}}^k$ such that for some nonzero tangent vector $\vec x\in\T_p$ the Jacobi operators $R^2_\vec x,\dots,R^k_\vec x$ have a common exceptional invariant subspace.

By Tarski--Seidenberg theorem, $\tilde\Sigma^k$ is a semialgebraic; in particular it is stratified.
Due to the Thom transversality theorem \cite[2.3.2]{eliashberg-mishachev}, it is sufficient to show that for any point $p$ the codimension of
$\tilde\Sigma^k$  in $\tilde{\mathcal G}^k$ is larger than $m=\dim M$.

This is a pointwise statement;
therefore we may fix $p$ from now on.

A jet in $\tilde{\mathcal{G}}^0$
is described by the metric tensor $g_0$  on $\T=\T_pM$.
Note that the forgetful map $\tilde{\mathcal{G}} ^k \to \tilde{\mathcal{G}}^0$ is a fiber bundle.
Furthermore the restriction of this forgetful map to $\tilde \Sigma$ is also a fiber bundle.
Thus it suffices to prove that the intersection $\Sigma^k$ of $\tilde \Sigma^k$ with a fiber of the forgetful map $\tilde{\mathcal{G}} ^k \to \tilde{\mathcal{G}}^0$
is a stratified subspace of the fiber of codimension at least $m$.
Note finally, that the fiber of the forgetful map over the  Euclidean structure
on $\T$ given by $g_0$ is exactly the space $\mathcal G^k$ investigated above.

In other words, if we choose a chart $\T\to M$, then $g_0$ defines an inclusion $\mathcal{G}^k\hookrightarrow\tilde{\mathcal{G}}^k$,
and it is sufficient to show that 
\[\codim\Sigma^k\to\infty
\quad\text{as}\quad
k\to \infty;
\eqlbl{eq:codim-Sigma}\]
here we consider $\Sigma^k\z=\tilde\Sigma^k\cap \mathcal{G}^k$ as a subset of $\mathcal{G}^k$.

Denote by $\mathcal{L}$ the semialgebraic set of all pairs $(L,\vec x)$ where $L$ is a subspace of $\T$ such that $1<\dim L<m$,  and $\vec x\in L\backslash \{0\}$.
Given $(L,\vec x)\in\mathcal{L}$, denote by $\Sigma^k(L,\vec x)$ the subset of jets in $\mathcal{G}^k$ such that $L$ is an invariant subspace of all Jacobi operators $R^i_\vec x$  for any $i\le k$.

Choose $(L,\vec x)\in\mathcal{L}$;
note that \ref{cor:Rall} and \ref{prop:submersion} imply that
\[\codim\Sigma^k(L,\vec x)\to\infty\quad \text{as}\quad k\to \infty.\eqlbl{eq:codim-Sigma(L,x)}\]
Indeed, a normal germ $(G^1,\dots G^k)$ belongs to $\Sigma^k(L,\vec x)$ if and only if all the Jacobi operators 
$R^2_\vec x,\dots, R^k_\vec x\in \mathcal{S}_\vec x$ have invariant subspace $L$.
By \ref{cor:Rall}, the codimension of 
the space of $(k-1)$-tuples in $S_x$ that has $L$ as invariant subspace grows with~$k$.
By \ref{prop:submersion} and \ref{cor:Rall} the composition 
$\mathcal{G}^k\z\to \mathcal{R}^k\z\to\mathcal{S}^{k-1}_\vec x$ that sends a germ to the array of its Jacobi operators
$(R^2_\vec x,\dots, R^k_\vec x)$ is a submersion.
Therefore \ref{eq:codim-Sigma(L,x)} follows.

Observe that 
\[\codim\Sigma^k\ge \codim\Sigma^k(L,\vec x)-\dim\mathcal{L}.\]
Therefore \ref{eq:codim-Sigma} follows.
\qeds

\section{Final remarks}\label{app:remarks}

It is expected that the following question admits an affirmative answer.

\begin{thm}{Open question}
Is it true that any Riemannian manifold $(M,g)$ contains a nontrivial geodesic  that runs in some convex hypersurface?
\end{thm}

It seems plausible that  the following  rigidity statement might be squeezed out from our Key Lemma, providing a more striking answer to Berger's question \cite[Note 1.6.3.1]{berger-2003}:

\begin{thm}{Conjecture}  Let $M$ be an arbitrary Riemannian manifold of dimension at least 3.
If the convex hull of any 3-point subset is compact
then $M$ has constant curvature.
\end{thm}

The following question clearly has a negative answer in dimension 2, possibly indicating  that also in higher dimensions the answer is negative.   

\begin{thm}{Open question}
Is it true that the set of maximal geodesics that run in a convex hypersurface of generic Riemannian manifolds is locally finite?
\end{thm}

The following question, related to the first one,  also seems to require new tools for a solution.

\begin{thm}{Open question}
Let $\mathfrak{C}$ be the closure of a convex hull of a set $Q$ in a Riemannian manifold.
Suppose $\gamma$ is a maximal $g$-geodesic that runs in $\partial \mathfrak{C}$.
Is it true that the ends of $\gamma$ belong to $Q$?
\end{thm}

The following question can be seen as a \emph{converse} statement of our main theorem.
Even for the Euclidean space it is a non-obvious
theorem proved by Anatoliy Milka \cite[§~4]{milka}.

\begin{thm}{Theorem}
Let $\mathfrak{C}$ be a closed convex set in an arbitrary $m$-dimensional Riemannian manifold $(M,g)$ and $\gamma$ be a geodesic in the boundary of $\mathfrak{C}$ with respect to the intrinsic metric of $\mathfrak {C}$.
Suppose that all points on $\gamma$ have rank at most $m-1$ in $\mathfrak{C}$.
Then $\gamma$ is a geodesic of the ambient space $(M,g)$.
\end{thm}

There is a chance that the presented argument, when properly extended to infinite-dimensional manifolds, might lead to a negative answer to the following question of Mikhael Gromov \cite[6.B\textsubscript{1}(f)]{gromov-1993}.

\begin{thm}{Open question}
Let $X$ be a complete length $\CAT(0)$ space (not necessary locally compact).
Is it true that any compact set of $X$ lies in a compact convex subset?
\end{thm}


{\sloppy
\printbibliography[heading=bibintoc]
\fussy
}
\end{document}
